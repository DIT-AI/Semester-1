
\documentclass[10pt,a5paper]{article}
\usepackage[ascii]{inputenc}
\usepackage{ngerman}
\usepackage{amsmath}
\usepackage{amsfonts}
\usepackage{amssymb}
\usepackage{graphicx}
\usepackage[left=1cm, right=1cm, top=1cm, bottom=1cm]{geometry}
\author{Christian B"ohm}



\newcommand{\sS}[1]{\textit{siehe Seite: \textbf{#1}}}
\renewcommand{\d}[1]{\Delta #1}


\begin{document}

\section{Grundlagen}

\newpage
\section{Kinematik}
\subsection{Eindimensionale Bewegungen}
\subsubsection{Gleichf{"o}rmige Translation}
\begin{eqnarray}
v=\frac{s}{t}
\end{eqnarray}
Umrechnung von Geschwindigkeiten: \sS{69}\\
Geschwindigkeiten in der Natur und Technik \sS{70}
\subsubsection{Gleichm{"a}{\ss}ig beschleunigte Translation}
\begin{eqnarray}
a=\frac{\d{v}}{\d{t}}
\end{eqnarray}
Beschleunigungen in Natur und Technik: \sS{71}\\
\textbf{Ohne Anfangsgeschwindigkeit}
\begin{eqnarray}
s=\frac{v_{end}*t}{2}\\
a=\frac{a*t^2}{2}\\
v=a*t\\
\overline{v}=\frac{a*t}{2}=\frac{s}{t}
\end{eqnarray}
 \textbf{Mit Anfangsgeschwindigkeit}
 \begin{eqnarray}
 s=\frac{v_0+v_{end}}{2}*t\\
 s=v_0*t+\frac{a*t^2}{2}\\
 v=v_0+a*t\\
 v=\sqrt{v_0^2+2as}\\
 \overline{v}=\frac{v_o+v_{end}}{2}=v_0+\frac{a*t}{2}=\frac{s}{t}\\
 \overline{v}=\frac{\d{s}}{\d{t}}=\frac{s_2-s_1}{t_2-t_1}\\
 s=s_0+\frac{v_0+v_{end}}{2}*t=s_0+v_0*t+\frac{a*t^2}{2}
 \end{eqnarray}
 \newpage
 \textbf{Ungleichm{"a}{\ss}ig beschleunigte Translation}
 \begin{eqnarray}
 v=\frac{ds}{dt}=\dot{s}\\
 s=\int_{t_1}^{t_2}v dt\\
 \overline{v}=\frac{s}{t}\\
 \overline{v}=\frac{\d{s}}{\d{t}}=\frac{s_2-s_1}{t_2-t_1}\\
 a=\frac{dv}{dt}=\ddot{s}\\
 v=\int_{t_1}^{t_2}adt\\
 \overline{a}=\frac{\d{v}}{\d{t}}=\frac{v_{end}-v_0}{t}\\
 \overline{a}=\frac{\d{v}}{\d{t}}\frac{v_2-v_1}{t_2-t_1}
 \end{eqnarray}
 \subsubsection{Fall und Wurf}
 \begin{eqnarray}
 h=\frac{v*t}{2}=\frac{g*t^2}{2}\\
 v=g*t=\sqrt{2*g*h}\\
 \end{eqnarray}
\textbf{Senkrechter Wurf:}
\begin{eqnarray}
h=\frac{v_0+v_{end}}{2}*t=v_0*t+\frac{g*t^2}{2}\\
v=v_0+g*t=\sqrt{v_0^2+2g*h}\\
h_{max}=\frac{v_0}{2*g}\\
t_{max}=\frac{v_0}{g}
\end{eqnarray}
\subsubsection{Zusammengesetze Bewegung}
\begin{eqnarray}
v_R=\sqrt{v_1^2+v_2^2+2*v_1*v_2*\cos\gamma}
\end{eqnarray}
\newpage
\subsection{Zweidimensionale Bewegungen}
\subsubsection{Waagrechter Wurf}
\begin{eqnarray}
y=\frac{g}{2*v_0}*x^2\\
\vec{r}=\vec{v_0}*t+\frac{\vec{g}*t^2}{2}\\
\vec{v_B}=\vec{v_0}+\vec{g}*t\\
\tan\gamma =\frac{g*t}{v_0}\\
v_B=\sqrt{v_0^2+g^2*t^2}\\
s=v_0*t=v_0\sqrt{\frac{2*h}{g}}\\
h=\frac{g*t^2}{2}
\end{eqnarray}
\subsubsection{Schr{"a}ger Wurf}
\begin{eqnarray}
y=x*\tan\gamma - \frac{g}{2*v_0^2*\cos^2\gamma}*x^2\\
\vec{v_x}=\vec{v_0}*\cos\gamma\\
\vec{v_y}=\vec{v_0}*\sin\gamma-\vec{g}*t\\
v_B=\sqrt{v_0^2-2*g*h}\\
s=v_0*t*\sin\gamma-\frac{g*t^2}{2}\\
t_{hmax}=\frac{v_0*\sin\gamma}{g}\\
t_{sm}=2*t_{hmax}=2*=\frac{v_0*\sin\gamma}{g}\\
h_{max}=\frac{v_0^2*\sin 2\gamma}{2*g}\\
s_{max}=\frac{v_0^2*\sin\gamma*\cos\gamma}{g}\\
\text{H{"o}henunterschied:}\hspace{9cm}\\
\text{nach max: }\hspace{1.5cm} 
h=h_{max}+\d{h};t_f=\sqrt{\frac{2*h}{g}};v_y=\sqrt{2*g*h}\\
\text{vor max: } \hspace{0.25cm}h=v_0*t*\sin\gamma-\frac{g*t^2}{2}\Rightarrow 
t;v_y=v_0*\sin\gamma-g*t
\end{eqnarray}
\newpage
\subsection{Rotation:}
\subsubsection{Grundlage}
\begin{eqnarray}
\phi=\frac{s}{r}\\
T=\frac{1}{f}=\frac{1}{n}\\
\phi=2\pi*N\\
\omega=2\pi*f=\frac{2\pi}{T} 
\end{eqnarray}
{"U}bersicht Winkel Rad/Grad: \sS{85}
\subsubsection{Gleichf�rmige Rotation}
\begin{eqnarray}
\omega=\frac{\phi}{t}
\end{eqnarray}
\subsubsection{Gleichm{"a}{\ss}ig beschleunigte Rotation}
\begin{eqnarray}
\alpha=\frac{\d{\omega}}{\d{t}}
\end{eqnarray}
\textbf{Ohne Anfangswinkelgeschwindigkeit}\\
\begin{eqnarray}
\phi=\frac{\omega*t}{2}=\frac{\alpha*t^2}{2}\\
\omega=\alpha*t=\sqrt{2\alpha *\phi}\\
\overline{\omega}=\frac{\alpha*t}{2}=\frac{\phi}{t}
\end{eqnarray}
\textbf{Mit Anfangswinkelgeschwindigkeit}\\
\begin{eqnarray}
\phi=\frac{\omega_0+\omega}{2}*t=\omega_0*t+\frac{\alpha *t^2}{2}\\
\phi_{ges}=\phi_0+\frac{\omega_0+\omega}{2}*t=\phi_0+\omega_0*t+\frac{\alpha*t^2}{2}
\end{eqnarray}\newpage
\subsubsection{Gleichm{"a}{\ss}ig beschleunigte Rotation}
\begin{eqnarray}
\omega=\frac{d\phi}{dt}\\
\phi=\int_{t_1}^{t_2}\omega dt\\
\overline{\omega}=\frac{\phi}{t}=\frac{\phi_2-\phi_1}{t_2-t_1}=\frac{\d{\phi}}{\d{t}}\\
\overline{n}=\overline{f}=\frac{\overline{\omega}}{2\pi}\\
\alpha=\frac{d\omega}{dt}=\dot{\omega}=\ddot{\phi}\\
\omega=\int_{t_1}^{t_2}\alpha dt\\
\overline{\alpha}=\frac{\d{\omega}}{\d{t}}=\frac{\omega-\omega_0}{t}=\frac{\omega_2-\omega_1}{t_2-t_1}
\end{eqnarray}
\textbf{Bewegung auf der Kreisbahn}\\
\begin{eqnarray}
s_B=\phi*r\\
v_B=\omega*r=d*\pi*f\\
a_B=\alpha*r\\
\vec{v_B}=\vec{\omega}\times\vec{r}\\
\vec{a_B}=\vec{\alpha}\times\vec{r}\\
a_r=\frac{v_B^2}{r}=\omega^2*r
\end{eqnarray}\newpage
\section{Dynamik}
\subsection{Masse und Kraft}
\subsubsection{Newton'sche Axiome}
\sS{98}
\subsubsection{Grundlagen}
\begin{eqnarray}
\vec{F}=\vec{a}*m\\
F_G=m*g\\
\end{eqnarray}
Umrechnung in SI Fremde Krafteinheiten: \sS{100}\\
\textbf{Dichte:}\\
\begin{eqnarray}
\rho=\frac{m}{V}\\
\rho_m=\frac{\sum_{i=1}^{n}\rho_i*V_i}{\sum_{i=1}^{n}V_i}
\end{eqnarray}
\textbf{Federkraft}\\
\begin{eqnarray}
k=\frac{F}{s}\\
F=-k*s\\
\end{eqnarray}
\textbf{Reibungskraft}\\
\begin{eqnarray}
F_R=\mu *F_N
\end{eqnarray}
\textbf{Tr{"a}gheit:}
\begin{eqnarray}
\vec{F_T}=-m*\vec{a}
\end{eqnarray}
\newpage
\subsubsection{Spezialf{"a}lle}
\vspace{5cm}
\begin{eqnarray}
\vec{F_{zH}}=\cos\gamma*F_z;\vec{F_{zV}=\sin\gamma*F_z}\\
\vec{F_N}=F_g-\vec{F_zV};\vec{F_R}=\mu*\vec{F_n}
\end{eqnarray}
\vspace{5cm}
\begin{eqnarray}
\vec{F_H}=\cos{90^\circ-\gamma}; 
\vec{F_n}=\cos\gamma*(F_g-\vec{F_{z\perp\gamma}})\\
\text{Tendenz nach oben: } 
\vec{F_{z\parallel\gamma}}=\vec{F_{R-H/G}}+\vec{F_H}\\
\text{Tendenz nach unten: } \vec{F_{z\parallel\gamma}}=\vec{F_H}\\
\text{Wenn }\vec{F_z}=\vec{F_{gk}}\\
\text{Tendenz nach oben: } m_k=m*(\mu_{G/H}*\cos\gamma+\cos(90^\circ-\gamma))\\
\text{Tendenz nach unten: } m_k=m*(\cos(90^\circ-\gamma)-\mu_{G/H}*\cos\gamma)
\end{eqnarray}
\newpage
\subsection{Arbeit, Energie und Leistung}
\subsubsection{Arbeit}
Umrechnung in SI-fremde Einheiten: \sS{106}\\
\begin{eqnarray}
W=F*s\\
W=F*s*\cos\gamma\\
W=\vec{F}*\vec{s}\\
W=\int_{s_1}^{s_2}F*\cos\gamma ds\\
W=\int_{s_1}^{s_2}\vec{F}*d\vec{s}
\end{eqnarray}
\textbf{Hubarbeit}\\
\begin{eqnarray}
W_H=F_G*h=m*g*h
\end{eqnarray}
\textbf{Reibungsarbeit}\\
\begin{eqnarray}
W_R=F_R*s=\mu*F_N*s\\
W=m*g*s+(\sin\gamma+\mu*\cos\gamma)
\end{eqnarray}
\textbf{Beschleunigungsarbeit}\\
\begin{eqnarray}
W_B=m*a*s=\frac{m*v^2}{2}\\
W_B=m*a*s=\frac{m}{2}(v^2-v_0^2)
\end{eqnarray}
\textbf{Verformungsarbeit}
\begin{eqnarray}
W_F=\frac{k*s^2}{s}
\end{eqnarray}
\newpage
\subsubsection{Energie}
\textbf{Potentielle}\\
\begin{eqnarray}
E_p=F_G*h=m*g*h\\
E_P=m*\int_{h_1}^{h_2}gdh
\end{eqnarray}
\textbf{Kinetische}\\
\begin{eqnarray}
E_k=\frac{m*v^2}{2}\\
\d{E_k}=\frac{m}{2}(v_2^2-v_1^2)\\
\end{eqnarray}
\textbf{Energieerhaltung}
\begin{eqnarray}
E_{ges}=E_p+E_k+E_r=konstant
\end{eqnarray}
\subsubsection{Leistung}
Umrechnung in SI-fremde Einheiten: \sS{114}\\
\begin{eqnarray}
\overline{P}=\frac{W}{t}\\
P=\frac{dW}{dt}=\dot{W}\\
P=F*v\\
\end{eqnarray}
\textbf{Wirkungsgrad}\\
\begin{eqnarray}
\eta=\frac{P_{ab}}{P_{zu}}=\frac{P_{zu}-P_{verlust}}{P_{zu}}=1-\frac{P_{verlust}}{P_{zu}}
\end{eqnarray}
\newpage
\subsection{Impuls und Sto{\ss}}
\subsubsection{Impuls}
\begin{eqnarray}
\vec{p}=m*\vec{v}
\end{eqnarray}
\subsubsection{Kraftsto{\ss}}
\begin{eqnarray}
\d{\vec{p}}=m*\d{\vec{v}}=\int_{t_1}^{t_2}\vec{F} dt\\
\vec{F}=\frac{d\vec{p}}{t}=\frac{d(m*\vec{v})}{dt}=\dot{\vec{p}}\\
\end{eqnarray}
\subsubsection{Impulssatz}
\begin{eqnarray}
\vec{p_ges}=\sum_{i=1}^{n}\vec{p_i}=konstant
\end{eqnarray}
\subsubsection{Sto{\ss}}
\textbf{Elastischer Sto{\ss}}\\
\begin{eqnarray}
v_1'=\frac{(m_1-m_2)*v_1+2*m_2*v_2}{m_1+m_2}\\
v_2'=\frac{(m_2-m_1)*v_2+2*m_1*v_1}{m_1+m_2}
\end{eqnarray}
\textbf{Unelastischer Sto{\ss}}\\
\begin{eqnarray}
v=\frac{m_1*v_1+m_2*v_2}{m1+m2}\\
W=E_1-E_2=\frac{m_1*m_2}{2*(m_1+m_2)(v_1-v_2)^2}
\end{eqnarray}
\textbf{Teilelastischer Sto{\ss}}\\
\begin{eqnarray}
\d{E}=\frac{m_1*m_2}{2*(m_1+m_2)}(v_1-v_2)^2(1-k^2)\\
v_1'=\frac{m_1*v_1+m_2*v_2-(v_1-v_2)m_2*k}{m_1+m_2}\\
v_1'=\frac{m_1*v_1+m_2*v_2+(v_1-v_2)m_1*k}{m_1+m_2}\\
k=\sqrt{\frac{h_2}{h_1}}
\end{eqnarray}
\newpage
\subsubsection{Spezialf{"a}lle}
\textbf{Ballistisches Pendel}\\
\begin{eqnarray}
E_{pot}=E_{kin}\\
v_{kh}=\sqrt{2*g*h}\\
v_{kh}=\frac{v_k*m_k}{m_k+m_h}=\sqrt{2*g*h}\\
v_k=\frac{m_h+m_k}{m_k}*\sqrt{2*g*h}
\end{eqnarray}
\newpage
\subsection{Drehbewegung}
\begin{eqnarray}
F_r=\frac{m*v^2}{r}=m*\omega^2*r=p*\omega\\
F_Z=\frac{m*v^2}{r}=m*\omega^2*r=p*\omega=-F_r\\
a_C=2*v*\omega\\
F_C=2*m*v*\omega\\
\vec{F_C}=2*m(\vec{v}\times\vec{\omega})\\
F_C=2*m*v*\omega*\sin\phi\\
\end{eqnarray}
\textbf{Arbeit bei der Rotation}\\
\begin{eqnarray}
W=M*\phi\\
W=\int_{\phi_1}^{\phi_2}Md\phi\\
\end{eqnarray}
\textbf{Leistung bei der Rotation}\\
\begin{eqnarray}
P=M\omega
\end{eqnarray}
\textbf{Rotationsenergie}\\
\begin{eqnarray}
E_{rot}=\frac{J*\omega^2}{2}\\
\d{E_{rot}}=\frac{J}{2}(\omega_2^2-\omega_1^2)
\end{eqnarray}
\textbf{Drehimpuls}\\
\begin{eqnarray}
\vec{L}=J\vec{\omega}\\
\vec{L}=\vec{r}\times\vec{p}=m*\vec{r}\times\vec{v}\\
\d{\vec{L}}=J\d{\vec{\omega}}=\vec{M}\d{t}=\int_{t_1}^{t_2}\vec{M} dt\\
\vec{M}=\frac{d\vec{L}}{dt}=\frac{d(J\vec{\omega})}{dt}=\dot{\vec{L}}\\
L_{ges}=\sum_{i=1}^{n}\vec{L_i}
\end{eqnarray}
\newpage
\subsection{Tr{"a}gheitsmoment}
Tr{"a}gheitsmoment \ensuremath{J_s} einiger K{"o}rper: \sS{131},\sS{132}
\begin{eqnarray}
M=J*\alpha\\
\vec{F}=J*\vec{\alpha}\\
J=r^2\d{m}=\sum_{i=1}^{n}r_i^2\d{m_i}\\
J=\int_{0}^{m_{ges}}r^2 dm=\rho\int_{0}^{V_{ges}}r^2 dV\\
J_S=m*r^2\\
J_A=J_S+m*s^2\\
m_{red}=\frac{J}{r^2}
i=\sqrt{\frac{J}{m}}\\
m*D^2=4J
\end{eqnarray}
Reduzierte Masse einiger K{"o}rper: \sS{134}\\
Tr{"a}gheitsradien eigener K{"o}rper: \sS{134} 
\newpage
\section{Gravitation}
Daten des Sonnensystem: \sS{148},\sS{149}\\
\begin{eqnarray}
F_G=\frac{m_1*m_2}{r^2}\\
g=g_0\frac{R^2}{r^2}=\frac{G*M}{r^2}\\
g_0*R^2=G*M\\
\vec{g}=\frac{\vec{F}}{m}\\
W=G*M*m(\frac{1}{r_1}-\frac{1}{r_2})\\
v_K=R\sqrt{\frac{g_o}{r}}=\sqrt{\frac{G*M}{r}}\\
r=\sqrt[3]{\frac{R^2*g_0*T^2}{4*\pi^2}}=\sqrt[3]{\frac{G*M*T^2}{4*\pi^2}}
\end{eqnarray}
Kepler'sche Gesetze: \sS{147f}
\newpage
\section{Schwingungen}
\begin{eqnarray}
\phi==\omega*t+\phi_0=2\pi*f+t+\phi_0\\
y=\hat{y}\sin\phi=\hat{y}\sin(\omega*t+\phi_0)\\
v=\hat{y}\omega\cos\phi=\hat{v}*\cos\phi\\
\hat{v}=\hat{y}\omega\\
a=-\hat{y}\omega^2\sin\phi=\hat{a}\sin\phi=-y\omega^2\\
\hat{a}=-\hat{y}\omega^2\\
F_R=-m\omega^2\hat{y}\sin\phi=m\hat{a}\sin\phi=-m\omega^2y\\
k=m\omega^2=-\frac{F_R}{y}\\
\ddot{y}+y\omega^2=0
\end{eqnarray}
\textbf{Lineare Federschwingung}\\
\begin{eqnarray}
k=\frac{F}{\d{l}}\\
\omega=\sqrt{\frac{k}{m}};f=\frac{1}{2\pi}\sqrt{\frac{k}{m}}; 
T=2\pi\sqrt{\frac{m}{k}}
\end{eqnarray}
\textbf{Drehschwingung}\\
\begin{eqnarray}
\epsilon=\hat{\epsilon}\sin\phi=\hat{e}\sin(\omega t+\phi_0)\\
\dot{e}=\hat{e}\omega\cos\phi=\hat{\dot{\epsilon}}\cos\phi\\
\ddot{\epsilon}=-\hat{\epsilon}\omega^2\sin\phi=-\hat{\ddot{\epsilon}}\sin\phi=-\epsilon\omega^2\\
\omega=\sqrt{\frac{D}{J}};f=\frac{1}{2\pi}\sqrt{\frac{D}{J}}; 
T=2\pi\sqrt{\frac{J}{D}}
\end{eqnarray}
\textbf{Pendelschwingungen}\\
\begin{eqnarray}
\text{Mathematisches Pendel}\\
T=2\pi\sqrt{\frac{l}{g}}\\
\text{Physisches Pendel}\\
T=2\pi\sqrt{\frac{J_A}{m*g*s}}\\
\text{Reduzierte Pendell{"a}nge}\\
l'=\frac{J_A}{m*g*s}\\
J_S=m*s*(\frac{gT^2}{4\pi^2}-s)
\end{eqnarray}
\textbf{Fl{"u}ssigkeitsschwingungen}\\
\begin{eqnarray}
T=2\pi\sqrt{\frac{l}{2g}}
\end{eqnarray}
\subsection{Schwingungsenergie}
\begin{eqnarray}
E=\frac{k*\hat{y}^2}{2}=\frac{m*\hat{v}^2}{2}
\end{eqnarray}
{"U}bersicht von Schwingungsenergi: \sS{205}
\newpage
\subsection{Freie ged{"a}mpfte Schwingung}
\begin{eqnarray}
\ddot{y}+2\delta\dot{y}+\omega_0^2=0\\
y=\hat{y}_0e^{-\delta t}\sin\phi\\
y=\frac{\hat{v}}{\omega_d}e^{-\delta t}\sin\omega_d *t\\
y=\hat{y_0}e^{-\delta 
*t}\frac{\omega_0}{\omega_d}cos(\omega_d*t-\arcsin\frac{\delta}{\omega_0})\\
\hat{y}_{i+n}=\frac{\hat{y_i}}{q^n}\\
e^{\delta T_d*n}=\frac{\hat{y}_i}{\hat{y}_{i+n}}=q^n\\
\Lambda=\delta T_d=\ln\frac{\hat{y}_i}{\hat{y}_{i+1}=\ln q}\\
\tau=\frac{1}{\delta}\\
T_{H}=\frac{\ln 2}{\delta}\\
\omega_d=\sqrt{\omega_0^2-\delta^2}=\omega_0\sqrt{1-\vartheta^2}\\
T_d=\frac{2\pi}{\sqrt{\omega_0^2-\delta^2}}=\frac{T_0}{\sqrt{1-\vartheta^2}}\\
y=\frac{\hat{v}}{\omega_d'}e^{-\delta t}\sinh\omega_d't\\
y=\hat{y}_0e^{-\delta t}\cosh\omega_d't\\
y=\hat{v}te^{-\delta t}\\
y=\hat{y}_0(1+\delta t)e^{-\delta t}
\end{eqnarray}
{"U}bersicht ged{"a}mpfte Schwingungen: \sS{211}
\newpage
\subsection{Erzwungene Schwingung}
\begin{eqnarray}
\ddot{y}+2\delta\dot{y}+\omega_0^2y=\frac{\hat{F}_E}{m}\cos\omega*t\\
y=\hat{y}\cos(\omega*t-\alpha)\\
\hat{y}=\frac{\hat{F_E}}{\sqrt{m^2(\omega_0^2-\omega^2)^2+\beta^2\omega^2}}\\
\alpha=\arctan\frac{\omega\beta}{m*(\omega_0^2-\omega^2)}=\arctan\frac{2\omega\delta}{\omega_0^2-\omega^2}\\
\hat{y}_{st}=\frac{\hat{F}_E}{m\omega_0^2}\\
\omega_R=\sqrt{\omega_0^2-\frac{\beta^2}{2m^2}}=\sqrt{\omega_0^2-2\delta^2}\\
\hat{y}=\frac{\hat{F}_E}{\beta\sqrt{\omega_0^2-\delta^2}}=\frac{\hat{F}_E}{\beta\omega_d}\\
\frac{\hat{y}_R}{\hat{y}_{st}}=\frac{\omega_0^2}{2\delta\sqrt{\omega_0^2-\delta^2}}=\frac{\omega_0^2}{2\delta\omega_d}=\frac{\pi\omega_0^2}{\Lambda\omega_d^2}\\
Q\approx\frac{\pi}{\Lambda}\\
\frac{\hat{y}}{\hat{y}_{st}}=\frac{\omega_0^2}{\sqrt{(\omega_0^2-\omega^2)^2+(2\delta\omega)^2}}\\
\end{eqnarray}
\newpage
\subsection{{"U}berlagerung von Schwingungen}
\subsubsection{gleiche Richtung und Frequenz}
\begin{eqnarray}
y_R=\hat{y}_R\sin(\omega*t+\phi_{0R})\\
\hat{y}_R=\sqrt{\hat{y}_1^2+\hat{y}_2^2+2\hat{y}_1\hat{y}_2\cos(\phi_{01}-\phi_{02})}\\
\phi_{0R}=\arctan\frac{\hat{y}_1\sin\phi_{01}+\hat{y}\sin\phi_{02}}{\hat{y}_1\cos\phi_{01}+\hat{y}\cos\phi_{02}}\\
\hat{y}_R=2\hat{y}_1\cos\frac{\phi_{01}-\phi_{02}}{2}\\
\phi_{0R}=\frac{\phi_{01}+\phi_{02}}{2}\\
\end{eqnarray}
\subsubsection{gleiche Richtung ungleiche Frequenz}
\begin{eqnarray}
y_R=2\hat{y}\cos(\frac{\omega_1-\omega_2}{2}*t)\sin(\frac{\omega_1+\omega_2}{2}*t)\\
f_s=f_1-f_2\\
T_s=\frac{T_1T_2}{T_2-T_1}\\
f_R=\frac{f_1+f_2}{2}=\overline{f}\\
T_R=\frac{2T_1T_2}{T_1+T_2}
\end{eqnarray}
\subsubsection{ungleiche Richtung}
\sS{222ff.}
\newpage
\section{Elektrische und Magnetische Felder}
\subsection{Magentisches Feld}
\vspace{5cm}
\begin{eqnarray}
\text{Coulubsches Gesetz: } 
\vec{F}=\frac{1}{4\pi\epsilon_0\epsilon_r}*\frac{Q_1*Q_p}{r^2}\\
\text{Feldst{"a}rke auf Probeladung: } \vec{E}=\frac{\vec{F}}{Q}\\
\text{Gravitation: } F=y\frac{m_1*m_2}{r^2}\\
\text{Spezifischer Widerstand: } R=\frac{1}{\kappa}*\frac{l}{A}\\
\text{L{"a}ngsysmetrie}\Rightarrow\text{mag. Quadrat bei U}\\
\text{FL{"a}chensymetrie}\Rightarrow\text{mag. Quadrat bei I}\\
\text{Grentfl{"a}chen im St{"o}rfeld: }\\
\text{Reihenschaltung: 
}\frac{E_1}{E_2}=\frac{\kappa_2}{\kappa_1}=\frac{\rho_1}{\rho_2}\\
\text{Parallelschaltung: 
}\frac{J_1}{J_2}=\frac{\kappa_1}{\kappa_2}=\frac{\rho_2}{\rho_1}\\
\text{Leistung im Str{"o}mungsfeld: 
}P=\int\vec{E}d\vec{s}*\int\int\kappa\vec{E}d\vec{A}\\
\text{Leistungsdichte: } p=\frac{P}{V}
\end{eqnarray}
\newpage
\subsection{Elektrisches Feld}
\vspace{5cm}
\begin{eqnarray}
\text{Flussdichte: } D=\frac{\Psi}{A}=\epsilon E\\
\text{Gau{\ss}scher Satz der Elektrostatik: 
}\Psi=\iint\vec{D}d\vec{A}=\iint\epsilon\vec{E}d\vec{A}\\
\text{Kapazit{"a}t: }C=\frac{\Psi}{U}; C=\frac{\epsilon_0\epsilon_r A}{d}\\
\text{Zylinderkondensator geschichtet: } C=\frac{2\pi 
l}{\ln(\prod_{i=2}^{n}(\frac{r_i}{r_{i-1}})^{\frac{1}{\epsilon_{i-1}}} }\\
\text{Verbrauchergesetzt des Kondensators: 
}i=C*\frac{du}{dt};\\
\text{Spannungstr{"a}gheit }u=\d{U}=\frac{1}{C}\int_0^t 
idt=\frac{q}{C}\\
\text{Reihenschaltung von Kondensatoren:}\\\text{Verhalten sich wie 
Widerst{"a}ande 
bei 
Paralellschlatung}\\
\text{Kapazitivier Spannungsteiler: 
}\frac{U_1}{U_2}=\frac{C_2}{C_1};\frac{U_2}{U_{ges}}=\frac{C_1}{C_1+C_2}\\
\text{Paralellschaltung von Kondensatoren:}\\\text{Verhalten sich wie 
Widerst{"a}nde in Reihe}\\
\text{Kapazitiver Stromteiler: 
}\frac{i_1}{i_2}=\frac{C_1}{C_2};\frac{i_2}{i_{ges}}=\frac{C_2}{C_1+C_2}
\end{eqnarray}
\newpage
\begin{eqnarray}
\text{Selbstenlatung: }\tau_{iso}=\frac{\epsilon_r\epsilon_0}{k}\\
\text{Geschichtetes Dielektrikum: }
\text{Reihenschaltung: } \frac{E_1}{E_2}=\frac{\epsilon_{r2}}{\epsilon_{r1}}\\
\text{Paralellschaltung: } \frac{D_1}{D_2}=\frac{\epsilon_{r1}}{\epsilon_{r2}}\\
\text{Gespeicherte Energie: 
}W=\frac{1}{2}*C*U^2=\frac{1}{2}*\frac{Q^2}{C}=\frac{1}{2}i_{max}*\tau*U\\
\text{Inhomogenes Feld: } W=\iint\epsilon\vec{E}d\vec{A}*\int \vec{E}d\vec{s}\\
\text{Kraft zwischen 2 Kondensatorplatten: } F_q=-\frac{1}{2}Q*E\\
\text{Spezielle Kondensatoren: }\\
\text{Sichtkondensator: } C=(n-1)*\frac{A*s*\epsilon}{d}\\
\text{Wickelkondensator: }C=2*\frac{\epsilon A}{d};A=l*b\\
\text{Drehkondensator: }C=(n-1)*\frac{\alpha}{\pi}*\frac{A*\epsilon}{d}
\end{eqnarray}
\textbf{Geladene Kondensatoren in Reihe}\\
Anfangswerte: \ensuremath{U_{1\alpha}=\frac{Q_{1\alpha}}{C_1}};\\ 
Gesamtspannung=Spannungszuwachs+Gesamtanfangsspannung: 
\ensuremath{u_g=\frac{1}{C_e}*\int_{0}^{t}idt=I_q*\d{t}}\\
Vorgehensweise bei einer Idealen Spannungsquelle:
\begin{itemize}
\item Brechung der Anfangsspannungen
\item Berechnung von der verschobenen Ladung
\item Entspannung am Kondensator
\item Kontrolle {"u}ber die obige Formel
\end{itemize}
Vorgehensweise bei idealer Spannungsquelle:
\begin{itemize}
\item Wirkspannung am ge{"o}ffneten Schalter berechnen
\item Verschobenen Ladung in dem Fall: \ensuremath{\d{Q}=C_e*U_w}
\item Spannung Kondensator: 
\ensuremath{U_{ie}=\d{U}+U_{ia}=\frac{Q_v}{C_1}+U_{ia}}
\end{itemize} 
\textbf{Geladene Kondensatoren Parallel}\\
Spannung an den Kondensatoren: \ensuremath{U_p=\frac{Q_ges}{C_ges}}, 
Anschlie�end neue Ladungsverteilung berechnen
\end{document}