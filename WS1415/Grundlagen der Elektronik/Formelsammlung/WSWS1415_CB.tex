
\documentclass[10pt,a5paper]{article}
\usepackage[latin5]{inputenc}
\usepackage{amsmath}
\usepackage{amsfonts}
\usepackage{amssymb}
\usepackage{graphicx}
\usepackage{ngerman}
\usepackage[left=0.00cm, right=0.00cm, top=0.00cm, bottom=0.00cm]{geometry}
\author{Christian Böhm}

\begin{document}

\section{Allgemeines}
\textbf{arithmetisches Mittel:} $\overline{a}=\frac{1}{T}\int_{t=0}^{T}a(t)dt$
\\\textbf{Gleichrichtwert:} $a_{GL}=|a(t)|=\frac{1}{T}\int_{t=0}^{T}|a(t)|dt$ Wechselgröße: $\overline{a}=0$\\
\textbf{Effektiv/RMS-wert:} $A=a_{eff}=\sqrt{\frac{1}{T}\int_{t=0}^{T}a(t)^2dt} $  
\\\textbf{Scheitel/Crestfaktor:} $k_s=\frac{\hat{a}}{a_{eff}}$ 
  \\\textbf{Kreisfrequenz:} $\omega=2\pi f=\frac{2\pi}{T}$
 \\\textbf{uLiCu:}
 \\\textbf{Impendanz:} $\overline{Z}=\frac{\hat{u}}{\hat{i}}$; \textbf{Resistanz:} $R=\frac{\hat{u}_{//}}{\hat{i}}=Z*cos\phi_{ui}$; \textbf{Reaktanz:} $X=\frac{\hat{u}_\bot}{\hat{i}}=Z*sin\phi_{ui}$; \\\textbf{R-Dreieck:} $Z^2=R^2+X^2$ 
 \\ \textbf{iCuLi:}
 \\ \textbf{Admintanz:} $Y=\frac{\hat{i}}{\hat{u}}=\frac{1}{Z}$; \textbf{Konduktanz:} $G=\frac{\hat{i}_{//}}{\hat{u}}=Y*\cos\phi_{ui}$; 
 \\\textbf{Suszeptanz:} $B=\frac{\hat{i}_\bot}{\hat{u}}=-Y*\sin\phi_{ui}$; 
 \\\textbf{Y-Dreieck:} $Y^2=G^2+B^2$ 
 
 \section{Zweipole} 
 \textbf{Ohmscher $\phi=0^\circ$:} $ Z_R=\frac{1}{Y_R}=\frac{\hat{u}}{\hat{i}}=R$; $R_R=\frac{1}{G_R}=\frac{\hat{u}_{//}}{\hat{i}}=R$; $X_R=\frac{1}{B_R}=\frac{\hat{u}_\bot}{\hat{i}}=0$\\ 
 \textbf{Induktiver $\phi=90^\circ$(UvI):} $Z_L=\frac{1}{Y_L}=\frac{\hat{u}}{\hat{i}}=\omega L$; $R_L=\frac{1}{G_L}=0$; $X_L=-\frac{1}{B_L}=\omega L$\\ 
 \textbf{Kapazitiver $\phi=-90^\circ$(IvU):} $Z_c=\frac{1}{Y_c}=\frac{1}{\omega C}$; $R_c=\frac{1}{G_c}=0$; $X_c=-\frac{1}{B_c}=-\frac{1}{\omega C}$
 \section{Schwingkreise}
 \subsection{Reihen}
 \textbf{Widerstand:} \ensuremath{\underline{Z}=R+\imath(\omega L-\frac{1}{\omega C})}; \ensuremath{Z_0=R}; \ensuremath{X_0=\sqrt{\frac{L}{C}}}; \textbf{Resonanzfrequenz:} \ensuremath{\omega_0=\frac{1}{\sqrt{L*C}}};
 \\\textbf{max Strom:} \ensuremath{I_{max}=I_0=\frac{U}{R}}; \textbf{Güte:} \ensuremath{Q=\frac{\omega_0*L}{R}=\frac{1}{\omega_0*R*C}=\frac{X_0}{R}=\frac{f_0}{B_f}=\frac{1}{R}*\sqrt{\frac{L}{C}}=\frac{f_0}{f_{go}-f_{gu}}}
 \\\textbf{Spannungserhöhung:} \ensuremath{U_{L0}=U_{C_0}=Q*U_{R0}=Q*U_q }
 \\\textbf{Betragsgang:} \ensuremath{Z(\omega)=\sqrt{R^2+(\omega L - \frac{1}{\omega C})^2}}; \textbf{Phasengang:} \ensuremath{\psi_Z(\omega)=\arctan{\frac{\omega L*\frac{1}{\omega C}}{R}}}
 \subsection{Parallel}
  \textbf{Leitwert:} \ensuremath{\underline{Y}=G+\imath(\omega C-\frac{1}{\omega L})}; \ensuremath{Y_0=G}; \ensuremath{B_0=\sqrt{\frac{C}{L}}}; \textbf{Resonanzfrequenz:} \ensuremath{\omega_0=\frac{1}{\sqrt{L*C}}};
  \\\textbf{max SPG:} \ensuremath{U_{max}=I_0=\frac{I}{G}}; \textbf{Güte:} \ensuremath{Q=\frac{\omega_0*C}{G}=\frac{1}{\omega_0*G*L}=\frac{B_0}{G}=\frac{f_0}{B_f}=\frac{1}{G}*\sqrt{\frac{C}{L}}=\frac{f_0}{f_{go}-f_{gu}}}
  \\\textbf{Stormerhöhung:} \ensuremath{I_{C0}=I_{L_0}=Q*I_{G0}=Q*I_q }
  \\\textbf{Betragsgang:} \ensuremath{Y(\omega)=\sqrt{G^2+(\omega C - \frac{1}{\omega L})^2}}; \textbf{Phasengang:} \ensuremath{\psi_Y(\omega)=\arctan{\frac{\omega C*\frac{1}{\omega L}}{G}}}
  \subsection{Gemeinsamem Kenngrößen}
  \textbf{Grenzfrequenzen:} \ensuremath{f_{go/gu}=f_0*(\sqrt{1+(\frac{1}{2Q})}\pm \frac{1}{2Q}); Q\gg 1\Rightarrow f_{go/gu}=f_0(1\pm \frac{1}{2Q})=f_0\pm \frac{B_f}{2}}
  \\\textbf{Dämpfung:} \ensuremath{d=\frac{1}{Q}}; \textbf{Leistung:} \ensuremath{P_{f_{go/gu}}=\frac{1}{2}P_{f_o}}; \textbf{Güte:} \ensuremath{Q=\frac{|Blindleistung(\omega_0)|}{Wirkleistung(\omega_o)}}\\\ensuremath{\frac{I(f_g)}{I_{max}}=\frac{U(f_g)}{U_{Max}=\frac{1}{\sqrt{2}}}}
  
  
 
 \newpage
 \section{Komplexe Rechnung $\imath^2=-1$}
  \textbf{\emph{Darstellung}}\\ \textbf{R-Form:} $\underline{A}=a_r+\imath*a_\imath=\Re(\underline{A})+\imath*\Im(\underline{A})$; 
  \\\textbf{P-Form:} $\underline{A}=A*e^{\imath*\alpha}$; 
  \\\textbf{Euler:} $e^{\imath\alpha}=\cos\alpha+\imath\sin\alpha$
  \\ \textbf{P in R:} $a_r=A*\cos\alpha$, $a_i=A*\sin\alpha$; 
  \\\textbf{R in P:} $A=\sqrt{a_r^2+a_i^2}$, $\alpha=\arctan\frac{a_i}{a_r}$ 
  \\\\\textbf{\emph{Rechnen}}
  \\ \textbf{konjugiert Komplex:} $\underline{A}^*=A*e^{-\imath\alpha}=a_r-\imath*a_\imath$; 
  \\\textbf{Addition:} $\underline{C}=a_r+b_r+\imath*(a_\imath+b_\imath)$; 
  \\\textbf{Subtraktion:} $\underline{C}=a_r-b_r+\imath*(a_\imath-b_\imath)$; 
  \\\textbf{Multiplikation:} $\underline{C}=A*B*e^{\imath*(\alpha+\beta)}$; 
  \\\textbf{Division:} $\underline{C}=\frac{A}{B}*e^{\imath*(\alpha-\beta)}$ 
  \\\textbf{Inversion:} $\underline{A}^{-1}=\frac{1}{\underline{A}}=\frac{1}{a_r+\imath*a_\imath}=\frac{a_r-\imath*a_\imath}{a_r^2-a_\imath^2}=\frac{1}{A}*e^{-\imath\alpha}$
  \\\\ \textbf{\emph{Komplexe Teiler }} %kapazitiver // Induktiver hinzufügen
;
 \\ \textbf{Stromteiler:} $\frac{\underline{I}_a}{\underline{I_e}}=\frac{\underline{Y}_2}{\underline{Y}_1+\underline{Y}_2}=\frac{\underline{Z}_1}{\underline{Z}_1+\underline{Z}_2}$
 \\ \textbf{Spannungsteiler:} $\frac{\underline{U}_a}{\underline{U_e}}=\frac{\underline{Z}_2}{\underline{Z}_1+\underline{Z}_2}=\frac{\underline{Y}_1}{\underline{Y}_1+\underline{Y}_2}$\\ 
  \\\textbf{\emph{Dreick Stern Umformung}} 
  \\\textbf{ $\Delta\to\bot$:} $\underline{Z}_A=\frac{\underline{Z}_{AB}*\underline{Z}_{AC}}{\underline{Z}_{AC}+\underline{Z}_{AB}+\underline{Z}_{BC}}=$; 
  \\\textbf{$\bot\to\Delta$:} $\underline{Z}_{AB}=\underline{Z}_A+\underline{Z}_B+\frac{\underline{Z}_A*\underline{Z}_B}{\underline{Z}_C}$
 
\section{Leistung}
 \textbf{Leistungsmomentanwert:} $p=u*i$, bei sinus-förmig $p=P-S*cos(2\omega t+\phi_U+\phi_I)$; 
 \\\textbf{Wirkleistung:} $P=U*I*\cos(\phi_{U-I})=S*\cos\phi_{U-I}$
 \\\textbf{Scheinleistung:} $S=U*I$; 
 \\\textbf{Blindleistung:} $Q=U*I*\sin\phi_{U-I}$; 
 \\\textbf{Leistungsfaktor:} $\lambda=\frac{P}{S}$; 
 \\\textbf{P-Dreieck:} $S^2=P^2+Q^2$;
 \\\textbf{Leistungsfaktor:} $\cos\phi=\frac{P}{S}$; $\tan\phi=\frac{Q}{P}$
 \subsection{Leistungsanpassung}
 \textbf{optimaler Anschlusswiderstand:} \ensuremath{Z_{aopt}=R_a+\imath X_i=R_i-\imath X_i=\underline{Z}^*_i}, \ensuremath{\underline{Z}'_v=\underline{Z}^*_i}
 \\\textbf{maximale Wirkleisung:} \ensuremath{P_{max}=\frac{U^2_q}{4*R_i}=\frac{U^2_q}{4*R_v}=\frac{|U^2_q|}{4*\Re\{\underline{Z}_i\}}=\frac{(Z*I)^2}{4*R_i}}
 \\\textbf{Wirkungsgrad:} \ensuremath{\eta=\frac{I^2*R_{opt}}{I^2*(R_i+R_{opt})}}
 \subsection{Betragsanpassung}
 \textbf{Widerstand:} \ensuremath{R_a=\sqrt{R^2_i+X^2_i}=|\underline{Z}_i|=Z_i}
 \\\textbf{Wirkleisung:} \ensuremath{P_a=\frac{U^2_q}{2*(Z_i+R_i)}=\frac{U^2_q}{2*(R_a+R_i)}}
 \newpage
 \section{Spezialf{"a}lle}
 \subsection{Belasteter Spannungsteiler}
 \begin{eqnarray}
 \frac{U_a}{U_e}=\frac{Z_{2L}}{Z_1+Z_{2L}}=\frac{Z_L*Z_2}{Z_1*(Z_L+Z_2)+Z_L*Z_2}=\frac{Y_1}{Y_1+Y_2+Y_L}
 \end{eqnarray}
 \subsection{Belasteter Stromteiler}
 \begin{eqnarray}
 \frac{I_a}{I_e}=\frac{Y_{2L}}{Y_1+Y_{2L}}=\frac{Y_L*Y_2}{Y_1*(Y_L+Y_2)+Y_L*Y_2}=\frac{Z_1}{Z_1+Z_2+Z_L}
 \end{eqnarray}
 \subsection{Reihe\ensuremath{\Leftrightarrow}Parallel}
 \subsubsection{Reihe}
 \begin{eqnarray}
 2(Z_1+Z_2)=Z_{par}
 \end{eqnarray}
 \subsubsection{Parallel}
 \begin{eqnarray}
 Z_{1\parallel 2}=\frac{Z_1*Z_2}{Z_1+Z_2}\\
 Z_{reihe1}=\Re\{Z_{1\parallel 2}\};Z_{reihe2}=\Im\{Z_{1\parallel 2}\}
 \end{eqnarray}
 \subsection{Komplexer Teiler}
 \subsubsection{Spannungsteiler}
 \begin{eqnarray}
 Z_1=\frac{U_e*Z_2}{U_a}-Z_2\\
 Y_1=\frac{-U_a*Y_2}{U_a-U_e}
 \end{eqnarray}
 \subsubsection{Stromteiler}
 \begin{eqnarray}
 Y_1=\frac{I_e*Y_2}{I_a}-Y_2\\
  Z_1=\frac{-I_a*Z_2}{I_a-I_e}
  \end{eqnarray}
\end{document}