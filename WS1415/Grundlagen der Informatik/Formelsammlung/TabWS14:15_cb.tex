\documentclass[10pt,a5paper]{report}
\usepackage[latin5]{inputenc}

    \usepackage{longtable}

\usepackage{amsmath}
\usepackage{amsfonts}
\usepackage{amssymb}
\usepackage{graphicx}
\usepackage[left=0.00cm, right=0.00cm, top=1.00cm, bottom=0.00cm]{geometry}
\author{Christian Böhm}
\begin{document}

\section{Datentypen}
\begin{tabular}{|c|c|c|c|}
\hline \rule[-2ex]{0pt}{5.5ex} Datentyp & Keyword & Größe in Bytes & 
Wertebereich \\ 
\hline \rule[-2ex]{0pt}{5.5ex} Zeichen & char & 1 & -128 bis 127 \\ 
\hline \rule[-2ex]{0pt}{5.5ex} ganze Zahl(kurz) & short (short int) & 2 & -32768 bis 32767 \\ 
\hline \rule[-2ex]{0pt}{5.5ex} ganze Zahl & int & 4(meist) & -2147483648 bis 2147483647 \\ 
\hline \rule[-2ex]{0pt}{5.5ex} Ganze Zahl lang & long (long int) & 4 & -2147483648 bis 2147483647 \\ 
\hline \rule[-2ex]{0pt}{5.5ex} ohne Vorzeichen & unsinged char & 1 & 0 bis 255 \\ 
\hline \rule[-2ex]{0pt}{5.5ex} ohne Vorzeichen & unsinged short & 2 & 0 bis 65535 \\ 
\hline \rule[-2ex]{0pt}{5.5ex} ohne Vorzeichen & unsinged int & 4(Meist) & 0 bis 4294967295  \\ 
\hline \rule[-2ex]{0pt}{5.5ex} ohne Vorzeichen & unsinged long & 4 & 0 bis 4294967295  \\ 
\hline \rule[-2ex]{0pt}{5.5ex} einfache Gleitkomma & float & 4 & Genauigkeit 7 Dezimale  \\ 
\hline \rule[-2ex]{0pt}{5.5ex} doppelte Gleitkomma & double & 8 & Genauigkeit 19 Dezimale  \\
\hline 
\end{tabular}
\newpage

\section{Ein/Ausgabe}
\begin{itemize}
\item Steuerzeichen:\begin{itemize}
 \item \textbackslash a: BEL - akustisches Warnsignal
 \item \textbackslash b: BS Backspace -Cursor um einen Position nach links
 \item \textbackslash f: FF formfeed - Seitenvorschub
 \item \textbackslash n: NL Newline- der Cursor geht zur nächsten Zeile
 \item \textbackslash r: CR Carriage return - der Cursor springt zum Anfang der Zeile
 \item \textbackslash t: HT Horiziontal tab - Zeilenvorschub zur nächsten 
 horizontalen Tabulatorposition
 \item \textbackslash v: VT vertical tab - Zeilenvorschub zur nächsten 
 vertikalen Tabulatorposition
 \item \textbackslash ": " wird ausgegeben
 \item \textbackslash ': ' wird ausgegeben
 \item \textbackslash ?: ? wird ausgegeben
 \item \textbackslash \textbackslash: \textbackslash wird ausgegeben
 \item \textbackslash 0: Endmakierung eines Strings
 \item \textbackslash nnn: Ausgabe eines Oktalwerts
 \item \textbackslash xhh: Augabe eines Hexadezimalwerts
\end{itemize}\newpage
\item Typ: \begin{itemize}
\item \%d \%i Dezimalzahl mit Vorzeichen
\item \%lld, \%lli Dezimalzahl mit Vorzeichen (long long)
\item \%o Oktalzahl
\item \%x \%X Hexadezimalzahl (klein/groß) (kein Vz)
\item \%llx, \%llX Hexadezimalzahl (kein Vz, long long)
\item \%u Dezimalzahl ohne Vorzeichen
\item \%llu Dezimalzahl ohne Vorzeichen (long long)
\item \%c Buchstabe(Charakter)
\item \%s Zeichenkette(String)
\item \%f Gleitkommazahl
\item \%e \%E Gleitkommazahl (Exponentialdarstellung)
\item \%g \%G Double (Exponentialdarstellung)
\item \%p Pointer
\item \%n Anzahl auszugebender Zeichen
\item \%a wie \%f (ab C99)
\item \%\% das Zeichen \%
\item \%[Bezeichner] Einlesen bis ein Zeichen eingegeben wurde was nicht in der Liste Bezeichner steht
\item \%[\ensuremath{\land}Bezeichner] Einlesen bis ein Zeichen eingegeben wurd was in der liste Bezeichner steht
\end{itemize}
\item Flagangabe, optionales "-"-Zeichen legt linksbÃŒndige Ausgabe fest "+"-Zeichen gibt Plus bei positiven zahlen aus
\item Breite, die Zahl Breite legt die minimales Breites des Ausgabefeldes fest
\item PrÀzision, legt die anzahl an Nachkommastellen fest
\end{itemize}
\newpage
\section{Trigraph-Zeichen}
\begin{tabular}{|c|c|}
\hline Trigraph-Zeichen & Zeichen \\ 
\hline ??= & \# \\ 
\hline ??( & [ \\ 
\hline ??) & ] \\ 
\hline ??/ & \textbackslash \\ 
\hline ??' & \ensuremath{\land} \\ 
\hline ??! & \ensuremath{\vert} \\ 
\hline ??< & \{ \\ 
\hline ??> & \} \\ 
\hline ??- & \textasciitilde \\ 
\hline 
\end{tabular} 
\newpage

\section{ASCII}

    \begin{longtable}{|c|c|c|c||c|c|c|c|}
    \hline
    Dez & Hex & Okt & Zeichen & Dez & Hex & Okt & Zeichen\\
    \hline  $ 0 $&$ 0x00 $&$ 000 $&$ NUL $&$ 32 $&$ 0x20 $&$ 040 $&$ SP$\\
       $ 1 $&$ 0x01 $&$ 001 $&$ SOH $&$ 33 $&$ 0x21 $&$ 041 $&$ ! $\\
       $ 2 $&$ 0x02 $&$ 002 $&$ STX $&$ 34 $&$ 0x22 $&$ 042 $&$ "'$\\
       $ 3 $&$ 0x03 $&$ 003 $&$ ETX $&$ 35 $&$ 0x23 $&$ 043 $&$ \# $\\
       $ 4 $&$ 0x04 $&$ 004 $&$ EOT $&$ 36 $&$ 0x24 $&$ 044 $&$ \$ $\\
       $ 5 $&$ 0x05 $&$ 005 $&$ ENQ $&$ 37 $&$ 0x25 $&$ 045 $&$ \% $\\
       $ 6 $&$ 0x06 $&$ 006 $&$ ACK $&$ 38 $&$ 0x26 $&$ 046 $&$ \& $\\
       $ 7 $&$ 0x07 $&$ 007 $&$ BEL $&$ 39 $&$ 0x27 $&$ 047 $&$ ' $\\
       $ 8 $&$ 0x08 $&$ 010 $&$ BS $&$ 40 $&$ 0x28 $&$ 050 $&$ (  $\\
       $ 9 $&$ 0x09 $&$ 011 $&$ TAB $&$ 41 $&$ 0x29 $&$ 051 $&$  ) $\\
       $ 10 $&$ 0x0A $&$ 012 $&$ LF $&$ 42 $&$ 0x2A $&$ 052 $&$ * $\\
       $ 11 $&$ 0x0B $&$ 013 $&$ VT $&$ 43 $&$ 0x2B $&$ 053 $&$ + $\\
       $ 12 $&$ 0x0C $&$ 014 $&$ FF $&$ 44 $&$ 0x2C $&$ 054 $&$ , $\\
       $ 13 $&$ 0x0D $&$ 015 $&$ CR $&$ 45 $&$ 0x2D $&$ 055 $&$ - $\\
       $ 14 $&$ 0x0E $&$ 016 $&$ SO $&$ 46 $&$ 0x2E $&$ 056 $&$ . $\\
       $ 15 $&$ 0x0F $&$ 017 $&$ SI $&$ 47 $&$ 0x2F $&$ 057 $&$ / $\\
       $ 16 $&$ 0x10 $&$ 020 $&$ DLE $&$ 48 $&$ 0x30 $&$ 060 $&$ 0 $\\
       $ 17 $&$ 0x11 $&$ 021 $&$ DC1 $&$ 49 $&$ 0x31 $&$ 061 $&$ 1 $\\
       $ 18 $&$ 0x12 $&$ 022 $&$ DC2 $&$ 50 $&$ 0x32 $&$ 062 $&$ 2 $\\
       $ 19 $&$ 0x13 $&$ 023 $&$ DC3 $&$ 51 $&$ 0x33 $&$ 063 $&$ 3 $\\
       $ 20 $&$ 0x14 $&$ 024 $&$ DC4 $&$ 52 $&$ 0x34 $&$ 064 $&$ 4 $\\
       $ 21 $&$ 0x15 $&$ 025 $&$ NAK $&$ 53 $&$ 0x35 $&$ 065 $&$ 5 $\\
       $ 22 $&$ 0x16 $&$ 026 $&$ SYN $&$ 54 $&$ 0x36 $&$ 066 $&$ 6 $\\
       $ 23 $&$ 0x17 $&$ 027 $&$ ETB $&$ 55 $&$ 0x37 $&$ 067 $&$ 7 $\\
       $ 24 $&$ 0x18 $&$ 030 $&$ CAN $&$ 56 $&$ 0x38 $&$ 070 $&$ 8 $\\
       $ 25 $&$ 0x19 $&$ 031 $&$ EM $&$ 57 $&$ 0x39 $&$ 071 $&$ 9 $\\
       $ 26 $&$ 0x1A $&$ 032 $&$ SUB $&$ 58 $&$ 0x3A $&$ 072 $&$ : $\\
       $ 27 $&$ 0x1B $&$ 033 $&$ ESC $&$ 59 $&$ 0x3B $&$ 073 $&$ ; $\\
       $ 28 $&$ 0x1C $&$ 034 $&$ FS $&$ 60 $&$ 0x3C $&$ 074 $&$ < $\\
       $ 29 $&$ 0x1D $&$ 035 $&$ GS $&$ 61 $&$ 0x3D $&$ 075 $&$ = $\\
       $ 30 $&$ 0x1E $&$ 036 $&$ RS $&$ 62 $&$ 0x3E $&$ 076 $&$ > $\\
       $ 31 $&$ 0x1F $&$ 037 $&$ US $&$ 63 $&$ 0x3F $&$ 077 $&$ ? $\\
    \hline
    \end{longtable}
    
\newpage
    \begin{longtable}{|c|c|c|c||c|c|c|c|}
    \hline
    Dez & Hex & Okt & Zeichen & Dez & Hex & Okt & Zeichen\\
    \hline
    64 & 0x40 & 100 & @ & 96 & 0x60 & 140 & ` \\
    65 & 0x41 & 101 & A & 97 & 0x61 & 141 & a \\
    66 & 0x42 & 102 & B & 98 & 0x62 & 142 & b \\
    67 & 0x43 & 103 & C & 99 & 0x63 & 143 & c \\
    68 & 0x44 & 104 & D & 100 & 0x64 & 144 & d \\
    69 & 0x45 & 105 & E & 101 & 0x65 & 145 & e \\
    70 & 0x46 & 106 & F & 102 & 0x66 & 146 & f \\
    71 & 0x47 & 107 & G & 103 & 0x67 & 147 & g \\
    72 & 0x48 & 110 & H & 104 & 0x68 & 150 & h \\
    73 & 0x49 & 111 & I & 105 & 0x69 & 151 & i \\
    74 & 0x4A & 112 & J & 106 & 0x6A & 152 & j \\
    75 & 0x4B & 113 & K & 107 & 0x6B & 153 & k \\
    76 & 0x4C & 114 & L & 108 & 0x6C & 154 & l \\
    77 & 0x4D & 115 & M & 109 & 0x6D & 155 & m \\
    78 & 0x4E & 116 & N & 110 & 0x6E & 156 & n \\
    79 & 0x4F & 117 & O & 111 & 0x6F & 157 & o \\
    80 & 0x50 & 120 & P & 112 & 0x70 & 160 & p \\
    81 & 0x51 & 121 & Q & 113 & 0x71 & 161 & q \\
    82 & 0x52 & 122 & R & 114 & 0x72 & 162 & r \\
    83 & 0x53 & 123 & S & 115 & 0x73 & 163 & s \\
    84 & 0x54 & 124 & T & 116 & 0x74 & 164 & t \\
    85 & 0x55 & 125 & U & 117 & 0x75 & 165 & u \\
    86 & 0x56 & 126 & V & 118 & 0x76 & 166 & v \\
    87 & 0x57 & 127 & W & 119 & 0x77 & 167 & w \\
    88 & 0x58 & 130 & X & 120 & 0x78 & 170 & x \\
    89 & 0x59 & 131 & Y & 121 & 0x79 & 171 & y \\
    90 & 0x5A & 132 & Z & 122 & 0x7A & 172 & z \\
    91 & 0x5B & 133 & [ & 123 & 0x7B & 173 & \{ \\
    92 & 0x5C & 134 & $\backslash$ & 124 & 0x7C & 174& $\vert$ \\
    93 & 0x5D & 135 & ] & 125 & 0x7D & 175 & \} \\
    94 & 0x5E & 136 & \^{} & 126 & 0x7E & 176 & \textasciitilde \\
    95 & 0x5F & 137 & \_ & 127 & 0x7F & 177 & DEL \\
    \hline
    \end{longtable}
    \newpage
    
    \begin{tabular}{|c|c|}
    \hline  & single  \\ 
    \hline Vorzeichen & 1 \\ 
    \hline $e_{min}$ & -126 \\ 
    \hline $e_{max}$ & 127 \\ 
    \hline Länge des Exponenten & 8 \\ 
    \hline Länge der Mantisse (Significant) & 24 \\ 
    \hline Gesamtlänge & 32 \\ 
    \hline Größte Darstellbare Zahl & $(1-2^{-24})*2^{128}=3,403*10^{38}$ \\ 
    \hline Kleinste normiter darstellbare $Zahl>0$ &$2^{-126}=1,175*10^{-38}$  
    \\ 
    \hline Kleinste nicht normiert darstellbare Zahl  
    &\ensuremath{2^{-149}=1,401*10^{-45}}\\ 
    \hline \ensuremath{\epsilon}(Abstand von 1 zur nächst größeren Zahl)  & 
    \ensuremath{2^{-23}=1,129*10^{-7}}  
    \\ 
\hline   
    \end{tabular} 
    \\
    \centering
     \begin{tabular}{|c|c|}
        \hline  & double \\ 
        \hline Vorzeichen & 1 \\ 
        \hline $e_{min}$ & -1022 \\ 
        \hline $e_{max}$ & 1023 \\ 
        \hline Länge des Exponenten & 11 \\ 
        \hline Länge der Mantisse (Significant) & 53 \\ 
        \hline Gesamtlänge & 64 \\ 
        \hline Größte Darstellbare Zahl & 
        $(1-2^{-53})*2^{1024}=1,789*10^{308}$ \\ 
        \hline Kleinste normiter darstellbare $Zahl>0$   
        &\ensuremath{2^{-1022}=2,225*10^{-308}} \\ 
        \hline Kleinste nicht normiert darstellbare Zahl  
        & \ensuremath{2^{-1074}=4,491*10^{-324}} \\ 
        \hline \ensuremath{\epsilon}(Abstand von 1 zur nächst größeren Zahl) & 
        \ensuremath{2^{-52}=2,220*10^{-16}}  
        \\ 
    \hline   
        \end{tabular} 
        
    \begin{figure}
\centering
\includegraphics[width=0.7\linewidth]{./754singel}
\end{figure}

\end{document}