
\documentclass[10pt,a5paper]{article}

\usepackage[latin5]{inputenc}                                        
\usepackage[T1]{fontenc}                                                 
\usepackage[ngerman,english]{babel}                                      
\usepackage{amsmath}                                                     
\usepackage{xifthen}                                                     
\usepackage{graphicx}                                                
\usepackage{units}                                                       
\usepackage{setspace}                                                    
\usepackage{hyperref}                                          
\usepackage[font=small,labelfont=bf,labelsep=endash,format=plain]{caption}    
\usepackage{subfig}                                                      
\usepackage{wrapfig}                                                     
\usepackage{cite}                                                       
\usepackage{scrpage2}                                                   
\usepackage{array,dcolumn}                                           
\usepackage{amssymb}
\usepackage{marvosym}
\usepackage{ngerman}


%KV
\input kvmacros 
%pstricks
\usepackage{pstricks}
\usepackage{pstricks-add}
\usepackage{graphicx}
\psset{xunit=1.0cm,yunit=1.0cm,algebraic=true,dimen=middle,dotstyle=o,dotsize=3pt 0,linewidth=0.8pt,arrowsize=3pt 2,arrowinset=0.25}

%commands
\newcommand{\cImage}[1]{
    \begin{figure}[h!]
        \centering
        \includegraphics[width=0.50\textwidth]{#1}
    \end{figure}
}  
\newcommand{\mm}[1]{\ensuremath{#1}}
\newcommand{\pg}[1]{\paragraph{#1} \textbf{ }\\}

%mathmodecommands 
\newcommand{\vI}[1]{\begin{pmatrix} #1\end{pmatrix}}
\newcommand{\vII}[2]{\begin{pmatrix} #1&#2\end{pmatrix}}
\newcommand{\vIII}[3]{\begin{pmatrix} #1&#2&#3\end{pmatrix}}
\newcommand{\vIV}[4]{\begin{pmatrix}#1&#2&#4\end{pmatrix}}
\newcommand{\ol}[1]{\overline{#1}}
\renewcommand{\vec}[1]{\overrightarrow{#1}}

\usepackage[left=0.00cm, right=1.00cm, top=1.00cm, bottom=0.00cm]{geometry}
\author{Christian Böhm}


\begin{document}
\section{Allgemein}
\textbf{Driftgeschwindigkeit:} \ensuremath{v=b*E}; \textbf{Strom:} \ensuremath{I=Q*n*b*A*E=\frac{dQ}{dt}};\\ \textbf{Einheit Leitfähigkeit:} \ensuremath{[k]=\frac{S}{m}=\frac{1}{\Omega m}}; \textbf{Spannung:} \ensuremath{U=\phi_1-\phi_2}
\section{Widerständen}
\subsection{Allgemein}
\textbf{Widerstand:} \ensuremath{R=\frac{U}{I}=\rho *\frac{l}{A}}
\subsection{Temperaturabhängigkeit}
\textbf{Lineare Näherung:} \ensuremath{R(\Theta)=R_{20}*(1+\alpha_{20}*\Delta\Theta) \text{ mit } \Delta\Theta=\Theta-20°C};\\
\textbf{Quadratische Näherung:} \ensuremath{R(\Theta)=R_{20}*(1+\alpha_{20}*\Delta\Theta+\beta_{20}*(\Delta\Theta)^2) \text{ mit } \Delta\Theta=\Theta-20°C};\\
\textbf{NTC:} Umso heißer das Material umso geringer ist der Widerstand;\\ \textbf{PTC:} Umso kälter das Material umso geringer ist der Widerstand 
\subsection{Leistung}
\textbf{Verbraucher Widerstand} \ensuremath{P=U*I=G*U^2=\frac{U^2}{R}=R*I^2}\\\textbf{Anpassung:} \ensuremath{R_a=R_i;\eta=0,5}
\subsection{Wirkungsgrad}
\ensuremath{\eta=\frac{R_V}{(R_V+R_ie )}}
\textbf{Maximale Leistung: }\ensuremath{P_abmax=(\frac{U_q^2)}{(4R_ie )}}\textbf{Quellenleistung: }\ensuremath{P_q=U_q*I}

\section{Teiler}
\textbf{Spannungsteiler:} \ensuremath{\alpha= \frac{U_a}{U_e}= \frac{R_2}{R_1+R_2 )} }\\
\textbf{Stromteiler: }\ensuremath{\beta=\frac{I_a}{I_E} =\frac{G_2}{(G_1+G_2 )}=\frac{R_1}{(R_1+R_2 )}}  
\section{Dreieck-Stern-Transformation}
 
\textbf{Dreieck \ensuremath{\Rightarrow} Stern: } \ensuremath{ R_A=\frac{ (R_B R_C) } { (R_A+R_B+R_C ) } }\\
\textbf{Stern \ensuremath{\Rightarrow} Dreieck: } \ensuremath{ G_A=\frac { (G_B G_C) } {(G_A+G_B+G_C ) } }


\newpage

\section{KPA}
\begin{itemize}
\item Hauptdiagonalelemente enthalten die Summen der an dem Knoten betroffenen Leitwerte 
\item Matrixelemente außerhalb der Hauptdiagonalen enthalten den negativen Leitwert zwischen den Knoten
\item Matrixelemente sind symmetrisch (1. Probe) 
\item Stromquellenvektor : Iq zum Knoten wird positiv gezählt
\item Vom Knoten weg wird Iq negativ gezählt 

\end{itemize}
\textbf{Spezialfälle}
\begin{itemize}
\item Reale Spannungsquellen werden durch reale Stromquellen ersetzt 
\item Ideale Spannungsquellen
\item Zunächst KPA ohne ideale Spannungsquelle aufstellen
\item \textbf{1:} Spannungszählpfeil zeigt von i nach j
\item \textbf{2:} Zeile i auf j addieren (auch bei Iq)
\item \textbf{3:} In der i Zeile: \ensuremath{g_{ii} = 1} \ensuremath{g_{ij}=-1} und der Rest 0
\item \textbf{4:} Zählpfeil von i auf 0: \ensuremath{g_{ii}} = 1 und der Rest 0
\item \textbf{5:} Zählpfeil von 0 auf i: Hilfsspannung \ensuremath{-U_q} und wie oben vorgehen
\item Bei größerem Superknoten wenn z.B. zwischen 3 Knoten ideale Spannungsquellen liegen werden alle Bereiche addiert und anschließend wie oben vorgegangen!

\end{itemize}
\section{Knoten+Maschenstromverfahren}
\subsection{Knotenregel}
\begin{eqnarray}
\sum_{N}^{i=1}I_i=0
\end{eqnarray}
\subsection{Maschenregel}
\begin{eqnarray}
\sum_{N}^{i=1}U_i=0
\end{eqnarray}

\end{document}