
\documentclass[10pt,a5paper]{article}
\usepackage[latin9]{inputenc}
\usepackage{amsmath}
\usepackage{amsfonts}
\usepackage{amssymb}
\usepackage{graphicx}
\usepackage[left=0.05cm, right=1cm, top=1cm, bottom=1cm]{geometry}
\author{Christian Böhm}
\usepackage{color}
\definecolor{grey}{RGB}{211,211,211}
\newcommand{\cbg}[1]{\colorbox{grey}{#1}}


\begin{document}
\section{Grundlagen}
\subsection{Eigenschaften}
 fest definierten und maschinell überprüfbaren Syntax (Struktur); eindeutige Semantik\subitem kein Interpretationsspielraum
\subsection{Compiler} Übersetzen eines Programms in Maschinensprache:
\begin{enumerate}
\item Scanner/Parser: feststellen der syntaktischen Korrektheit
\item Codegenerator: Erzeugen des Maschinencodes ggf. Optimierung
\end{enumerate} 
\subsection{Interpreter}
Alternative zum Compiler
\begin{itemize}
\item QT wird bei jeder Programmausführung gelesen
\item Interpretiert Computerprogramm (Pro Befehl eine Befehlsfolge) schrittweise
\end{itemize}
\subsection{Linker}
\begin{itemize}
\item Teil der Compilers
\item Baut Programm aus verschiedenen Teilen zusammen
\item Ergänzt die vom Compiler erzeugen Maschinencodes um vordefinierte Bibliotheken
\item ermöglicht getrennte und verteilte Entwicklung
\end{itemize}
\subsection{Debugger}
Werkzeug zur Analyse und Behebung von Fehlern
\newpage \section{C}
\subsection{Geschichte}
\begin{itemize}
\item 1971-1973 von Dennis Ritchie in den Bell Laboratories fÃŒr die Programmierung des damals neuen UNIXBetriebssystems entwickelt.
\item Basis: Programmiersprache B von Ken Thompson in den Jahren 1969/70 geschrieben.
\item B wiederum geht auf die von Martin Richards Mitte der 1960er-Jahre entwickelte Programmiersprache BCPL zurÃŒck. 
\item 1973 erster Unix-Kernel (Betriebbsystem) in C geschrieben 1978 veröffentlichen Brian W. Kernighan und Dennis Ritchie die erste Auflage von "The C Programming Language" deutsch: Programmieren in C).Die darin beschriebene Fassung von C, die nach den Buchautoren "K\&R C" genannt wird, wird die erste  informelle C Referenz.

\item C verbreitete sich rasch und wird laufend weiterentwickelt:
 \begin{itemize}
\item Normierung durch ANSI 1989 (ANSI X3.159-1989 Programming Language C)
\item 1990 entsprechende ISO Norm C90 (mit kleinen Änderungen als C90
\item 1995 ISO Norm (C95) und 1999 ISO/IEC 9899 (C99).

\end{itemize}
\end{itemize}
\section{Programmiersprache C}
\begin{itemize}
\item Haupteinsatzgebiete: \begin{itemize}
\item Systemprogrammierung
\item Betriebssysteme (Hardwarenahe Programmierung zB Treibe)
\item Eingebettete Systeme
\item hochperformante Programmteile oder Applikationen
\end{itemize}
\item Basis/Grundlage vieler Andere Programmiersprachen
\end{itemize}
\textbf{Programmkomponenten}

\begin{itemize}
\item Grundstruktur eines C Programms
\item Variabeln
\item Konstanten
\item Datentypen
\end{itemize}

\newpage 
\subsection{Grundstruktur eines C-Programms}
 \begin{itemize}
 \item Anweisungen in Einzelschritten (max. eine Anweisung pro Zeile)
 \item Anweisung durch Semikolon abgeschlossen
 \item Zusammengehörende Anweisungen werden durch \{\dots\} zusammengefasst
 \item Textkonstanten werden durch "\dots" begrenzt
 \item Kommentare werden durch /*\dots*/ begrenzt bzw. //\dots bis Zeilenende
 \item mit \cbg{getchar();} wird auf Tastendruck gewartet, WICHTIG: Eingabepuffer mit \cbg{fflush(stdin);} leeren
 \end{itemize}
\subsubsection{Main Funktion}
\begin{itemize}
\item Bestandteil von jedem C Programm
\item Start bzw. Endpunkt vom Hauptprogramm
\item mit "\{\dots\} " abgegrenzt
\item kein komplettes Programm
\end{itemize}
\subsubsection{Präprozessoranweisungen }
\begin{itemize}
\item werden mit \# eingeleitet
\end{itemize}
\subsubsection{Variablen}
\begin{itemize}
\item Identifikatoren\begin{itemize}
\item nur Buchstaben, Ziffern und \_ 
\item erstes Zeichen darf keine Ziffer
\item Groß- und Kleinschreibung wird unterschieden
\item C-Keywords dÃŒrfen nicht verwendet werden
\item SelbsterklÀrend
\item Variablen im CamelCase oder mit underscore
\item Temporaries dem Typen angepasst
\item Englische Namen
\end{itemize}
\item Typisierung \begin{itemize}
\item Typ gibt den Speicher an
\item Typen sind vordefiniert
\item Typen können in Programm definiert werden
\end{itemize}
\item Deklaration: typ variabelname; \begin{itemize}
\item fast überall definierbar
\item legt Variabel an und macht sie Bekannt
\item nach Deklaration ist Var in Block (\{\dots\}) bekannt
\end{itemize}
\item sind immer zu initialisieren
\end{itemize}
\subsubsection{Konstanten}
\begin{itemize}
\item Literalkonstanten \begin{itemize}
\item dezimale Zahl (1,2,3)
\item hexadezimale Zahl, PrÀfix 0x (0x1,0x2,0x3)
\item oktale Zahl, PrÀfix 0 (010)
\item Gleitkomma (10.5,3.14)
\end{itemize}
\item Konstanten \begin{itemize}
\item Variablen auch als Konstanten
\item mit \cbg{const} initialisiert
\item nicht zu Überschreiben, werden im Speicher abgelegt
\item auch als Präprozessor definierbar \cbg{\#define KONSTANTENAME Ausdruck} wird dann Textuell ersetzt (keine Speicherung im Programm)

\end{itemize}
\end{itemize}
\subsubsection{Datentypen}
\begin{tabular}{|c|c|c|c|}
\hline \rule[-2ex]{0pt}{5.5ex} Datentyp & Keyword & Größe in Bytes & Wertebereich \\ 
\hline \rule[-2ex]{0pt}{5.5ex} Zeichen & char & 1 & -128 bis 127 \\ 
\hline \rule[-2ex]{0pt}{5.5ex} ganze Zahl(kurz) & short (short int) & 2 & -32768 bis 32767 \\ 
\hline \rule[-2ex]{0pt}{5.5ex} ganze Zahl & int & 4(meist) & -2147483648 bis 2147483647 \\ 
\hline \rule[-2ex]{0pt}{5.5ex} Ganze Zahl lang & long (long int) & 4 & -2147483648 bis 2147483647 \\ 
\hline \rule[-2ex]{0pt}{5.5ex} ohne Vorzeichen & unsinged char & 1 & 0 bis 255 \\ 
\hline \rule[-2ex]{0pt}{5.5ex} ohne Vorzeichen & unsinged short & 2 & 0 bis 65535 \\ 
\hline \rule[-2ex]{0pt}{5.5ex} ohne Vorzeichen & unsinged int & 4(Meist) & 0 bis 4294967295  \\ 
\hline \rule[-2ex]{0pt}{5.5ex} ohne Vorzeichen & unsinged long & 4 & 0 bis 4294967295  \\ 
\hline \rule[-2ex]{0pt}{5.5ex} einfache Gleitkomma & float & 4 & Genauigkeit 7 Dezimale  \\ 
\hline \rule[-2ex]{0pt}{5.5ex} doppelte Gleitkomma & double & 8 & Genauigkeit 19 Dezimale  \\
\hline 
\end{tabular}
\textbf{Typ-KompatibilitÀt}
\begin{itemize}
\item Standardtypen sind zueinander Kompatibel, können mit Operationen verknüpft werden (kann Nebeneffekte haben)
\item char hat einen "Doppelrolle" sowohl Zeichen als auch Zahl
\item Fehlerquellen: \begin{itemize}
\item Abschneiden des Nachkommateils
\item Abscheiden der höherwertigen Bits oder Rundungsfehler
\item Verlust des Vorzeichenes, Anderung des Wertes
\end{itemize}
\end{itemize}
 \newpage
\subsubsection{Ein/Ausgabe}
\textbf{Ausgabe:} \cbg{printf(\ensuremath{\text{"}< \%[Flag][Breite][\text{.PrÀzision}][\text{PrÀfix}]Typ>\text{"}},\ensuremath{<auszugebende Daten>} );}
\begin{itemize}
\item Steuerzeichen:\begin{itemize}
 \item \textbackslash a: BEL - akustisches Warnsignal
 \item \textbackslash b: BS Backspace -Cursor um einen Position nach links
 \item \textbackslash f: FF formfeed - Seitenvorschub
 \item \textbackslash n: NL Newline- der Cursor geht zur nÀchsten Zeile
 \item \textbackslash r: CR Carriage return - der Cursor springt zum Anfang der Zeile
 \item \textbackslash t: HT Horiziontal tab - Zeilenvorschub zur nÀchte horizontalen Tabulatorposition
 \item \textbackslash v: VT vertical tab - Zeilenvorschub zur nÀchte vertikalen Tabulatorposition
 \item \textbackslash ": " wird ausgegeben
 \item \textbackslash ': ' wird ausgegeben
 \item \textbackslash ?: ? wird ausgegeben
 \item \textbackslash \textbackslash: \textbackslash wird ausgegeben
 \item \textbackslash 0: Endmakierung eines Strings
 \item \textbackslash nnn: Ausgabe eines Oktalwerts
 \item \textbackslash xhh: Augabe eines Hexadezimalwerts
\end{itemize}
\item Typ: \begin{itemize}
\item \%d \%i Dezimalzahl mit Vorzeichen
\item \%o Oktalzahl
\item \%x \%X Hexadezimalzahl (klein/groß)
\item \%u Dezimalzahl ohne Vorzeichen
\item \%c Buchstabe(Charakter)
\item \%s Zeichenkette(String)
\item \%f Gleitkommazahl
\item \%e \%E Gleitkommazahl (Exponentialdarstellung)
\item \%g \%G Double (Exponentialdarstellung)
\item \%p Pointer
\item \%n Anzahl auszugebender Zeichen
\item \%a wie \%f (ab C99)
\end{itemize}
\item Flagangabe, optionales "-"-Zeichen legt linksbündige Ausgabe fest "+"-Zeichen gibt Plus bei positiven zahlen aus
\item Breite, die Zahl Breite legt die minimales Breites des Ausgabefeldes fest
\item Präzision, legt die anzahl an Nachkommastellen fest
\end{itemize}
\textbf{Eingabe:} \cbg{scanf(\ensuremath{\text{"}< \%[Flag][Breite][.\text{PrÀzision}][\text{PrÀfix}]Typ>\text{"},<Adresse>});}
\subsubsection{Operatoren}
\begin{itemize}
\item Verknüpfung/Manipulation von Daten
\item Verwendbar bei Variablen Konstante und Ausdrücken
\item Allgemeine Operatoren:\begin{itemize}
\item = Zuweisungsoperator
\item * Multiplikation, / Division, + Addition, - Subtraktion
\item \& Adressoperator
\item \% Modulo (Restbildung)
\item == Gleichheit 
\item != Ungleichheit
\item   \ensuremath{<,<=,>=,>} Vergleich auf kleiner/größer
\item  \&\& logisches und, \ensuremath{\vert} logisches oder, ! Negation 
\end{itemize}
\item Bitweise Operatoren: \begin{itemize}
\item \& und, \ensuremath{\land}  oder,  \ensuremath{\sim}  Negation
\item ? Bedingter Ausdruck \subitem \ensuremath{<boolscher Ausdruck>}?TrueStatement:Falsestatement\item Bitweise Operatoren verknÃŒpfen die einzelnen Bits 

\end{itemize}
\item Schiebe Operatoren(shifts): \begin{itemize}
\item \ensuremath{<< \text{ Linksshift, } >> \text{ Rechtsshift}}
\item Linksshift ziehen immer einen "0" nach, Rechtsshift auf unsinged Größen ebenfalls
\item Rechtsshift auf auf singed Größen ziehen das Vorzeichenbit nach
\item Multiplikation bzw. Division durch 2er Potenzen (bei Laufzeitkritischen Anwendungen)
\end{itemize}
\item einstellige Operatoren:\begin{itemize}
\item \ensuremath{-} Negation
\item \ensuremath{++} Inkrement \ensuremath{i++\widehat{=} i=i+1}
\item \ensuremath{--} Dekrement \ensuremath{i--\widehat{=} i=i-1}
\item \ensuremath{+=n} Addition einer Zahl n \ensuremath{x+=b\widehat{=} x=x+b} (analog \ensuremath{*=;/=;\% =})
\end{itemize}
\end{itemize}
\subsubsection{Kontrollstrukturen}
\begin{itemize}
\item if \dots else \begin{itemize}
\item Syntax: if (<Bedingung>) \{\ensuremath{<TrueStatement>}\} else \{\ensuremath{<Falsestatement>}\}
\item Anwendung: Vergleiche
\end{itemize}
\item switch \dots case \dots\begin{itemize}
\item Syntax: switch(Wert) \{case Wert\textsuperscript{i}: \ensuremath{<Anweisungsblock>}; break; default:\ensuremath{ <Anweisungsblock>} \} (case Wert\textsuperscript{i} müssen Konstanten sein)
\item Anwendung: Überschaubare Anzahl diskreter Werte
\end{itemize}
\item while \dots\begin{itemize}
\item Syntax: while(Bedingung) \{\ensuremath{<Anweisungsblock>}\}
\item Ausführung nur wenn Bedingung wahr
\end{itemize}
\item do \dots while \dots\begin{itemize}
\item Syntax: do \ensuremath{<Anweisungsblock>} while (Bedingung)
\item Ausführung min einmal auch wenn Bedingung falsch (While schleife Bevorzugen)
\end{itemize}
\item for \dots\begin{itemize}
\item Syntax: \subitem for\ensuremath{(<Start>}; \ensuremath{<Abbruchbedingung>}; \ensuremath{ <\text{Änderung}> )} \ensuremath{<Anweisungsblock>}
\end{itemize}

\item goto\begin{itemize}
\item Syntax: goto Sprungmarke; (Sprungmarke Setzen\ensuremath{<Sprungmarke>:} )
\item fÃŒhrt zu unÃŒbersichtlichem Code, verboten
\end{itemize}
\item break: Beendet Schleifenbearbeitung, Fortsetzung nach der Schleife
\item continue: Beendet Aktuelle SchleifenausfÃŒhrung, es wird mit der AusfÃŒhrung der Bedingung fortgesetzt
\item return: Ende einer Funktionsberechnung inkl. RÃŒckgabewert
\item exit: beendet ProgrammausfÃŒhrung
\end{itemize}
\subsubsection{Arrays und Strings}
\begin{itemize}
\item Deklaration:
\subitem\ensuremath{<typ>}varname[\ensuremath{<ganzzahliger Ausdruck>}];
\item Globaler Zugriff:
\subitem variabelname, Individueller Zugriff: varaibelname[\ensuremath{<ganzzahliger Ausdruck>}]
\item erstes Element hat Index -1, letztes Element hat Index \subitem Anzahl der Vektorelemente -1
\item Array = Array nicht möglich
\item automatische Ermittlung der Elemente: type varname[ ], Beachte:Strings werden immer durch Null-Bit beendet ("\textbackslash 0")
\item zur Manipulation: \cbg{\#include\ensuremath{<string.h>}}\begin{itemize}
\item scanf("\%s",\dots) Einlesen Zeichenkette
\item printf("\%s",\dots) Ausgeben von Zeichenketten
\item gets(\dots) Einlesen von Zeichenketten
\item puts(\dots) Ausgeben von Zeichenketten
\item \dots = getchar() Einlesen von Buchstaben
\item char* strchr(str,Zeichen)  suchen nach "Zeichen" in "str" \subitem Null wenn nicht vorhanden, sonst Adresse des ersten Zeichens
\item strcpy(zielstr, quellstr)  strlÀnge muss passen
\item strncpy(char *dest, const char *src, size\textunderscore t n);
\item strcat(zielstr, quellstr)  quell an zielstr anhÀnge strLÀnge von zielstr muss passen
\item strncat(char *dest, const char *src, size\textunderscore t n )
\item strcmp(str1,str2) vergleicht str1 mit str2 (alphabetisch) 0 wenn gleich; \ensuremath{<0} falls \ensuremath{str1<str2}; \ensuremath{>0} falls \ensuremath{str1>str2}
\item strncmp(const char *s1, const char *s2, size\textunderscore t n) 
\item int strcspn(str1,str2) Index des Ersten Zeiches von STR1, welches in Str2 vorkommt
\item int strlen(str) LÀnge von str ohne Stringende
\item char strstr(str1, str2): Festestellen wo str2 in str1 \subitem Ergebnis ist Zeiger auf den Teilstring in str1 falls str2 in str1 vorkommt sonst auf 0
\end{itemize}
\item Mehrdimensionale Array:\begin{itemize}
\item Deklaration:
\subitem\ensuremath{<typ>}varname[\ensuremath{<Zeile>}][\ensuremath{<Spalte>}]=\{\dots \}\dots \{\dots \}   (Zeile mit "\{\}" abgetrennt Splaten mit ",")
\item Zugrif auf die Elemente: varname[Zeile][Spalte]
\end{itemize}
\end{itemize}
\subsubsection{Benutzerdefinierte Datentypen}
\begin{itemize}
\item typedef Typdefinition Neuer\textunderscore Typ; (Standarttypen neu Definieren)
\item enum {\dots}; (Durchnummerieren der Inhalte neue Nummerierung mit "inhalt=nummer")
\item structs\begin{itemize}
\item Zusammenbau eines neuen Typs aus bestehenden Typen
\item typedef struct struct\textunderscore name {typ komponentename} typname;
\item Deklaration typname varname;, Komponenten Zugriff varname.komponentename
\end{itemize}
\item union \begin{itemize}
\item typedef union n\textunderscore name \{typ varname\} n; n\textunderscore i
\item die Komponenten belegen den selber Speicherplatz und haben den Gleichen wert, dieser wird nur anders interpretiert
\end{itemize}
\end{itemize}
\subsubsection{Präprozessor}
\begin{itemize}
\item Datenverarbeitung rein Textuell, keine Beachtung des Syntax
\item Direktiven \begin{itemize}
\item \#include\ensuremath{<datei>} sucht nach datei im system-verzeichnissen; \#include"\ensuremath{datei}" sucht in den Quellverzeichnissen
\item \#define TEXT Ersatztext ersetzt Text durch Ersatztext
\subitem \#define Makroname(Parameter\textunderscore i) Ersatztext (Parameter\textunderscore i wird durchen den I-ten parameter ersetzt)
\item \#if AUSDRUCK (Wenn Ausdruck True wird Code übernommen)
\item \#ifdef Konstante (Ausdruck wird nur übernommen wenn Konstante Definiert)
\item \#ifndef Konstante (Ausdruck wird nur übernommen wenn konstante nicht definiert)
\item \#else (Alternative zu vorhergehender \#if)
\item \#endif (schließt \#if)
\end{itemize}
\item symbolische Konstante\begin{itemize}
\item \textunderscore\textunderscore LINE\textunderscore\textunderscore
\item \textunderscore\textunderscore FILE\textunderscore\textunderscore
\item \textunderscore\textunderscore DATE\textunderscore\textunderscore
\item \textunderscore\textunderscore TIME\textunderscore\textunderscore
\end{itemize}
\end{itemize}
\subsubsection{Pointer}
\begin{itemize}
\item Ein Pointer zeigt auf eine Stelle im Speicher (Physikalische Speicheradresse) und kann eine Konstante(selten außer Null-Pointer) oder eine Variabel sein und ist typisiert
\item Eine Pointervariabel beinhaltet die Speicheradresse und steht selbst im Speicher
\item Pointer können auch auf Dateien oder Funktionen zeigen und werden als Referenz bezeichnet
\item Verwendung:\begin{itemize}
\item Hardwarenahe Programierung (Ansprechen von Registern, Ports, Interrupttabellen)
\item Resultat/Parameter Übergabe
\item dynamische Datenstrukturen (Listen/Bäume)
\end{itemize}
\item Deklaration: typ *Pointername; (* drückt aus das es ein Pointer ist)
\item typ p;(p ist ein typ) - typ *p;(p ist ein pointer auf einen typ) - typ **p;(p ist ein pointer auf einen pointer auf einen typ)
\item Pointer als Ergebnis einer Funktion: typ *f(typ1 p1,\dots)
\item Pointer als Parameter einer Funktion: ergtyo f(typ *p,\dots)
\item Pointer in funktionen als Parameter/Ergebnis erspart das Kopieren großer Datenmengen, ohne Alternative bei Dynamischen Strukturen
\item Dereferenzieren:\begin{itemize}
\item zugriff auf den speicher auf den der Pointer zeigt
\item liefert lesend einen wert von typ auf den der pointer zeigt
\item liefert schreiben eine Speicherstelle vom Typ auf der der Pointer zeigt
\item dieser wert btw die speicherstelle können in operationen oder als parameter oder als ergebnis einer funktion verarbeited werden
\item Dereferenzieren: *pointer wird in einem ausdruck verwendet (muss mit der typ der pointer kompatibel sein und muss immer auf eine bestimmte Speicheradresse zeigen)
\end{itemize}
\item Struckts\begin{itemize}
\item typedef struct{\dots} struct\textunderscore name; struct\textunderscore name *spointer; struct\textunderscore name s;spointer= \&s
\item Zugriff:\subitem (*spointer).\dots = inhalt\subitem spointer\ensuremath{->}\dots = inhalt
\end{itemize}
\item Arrays\begin{itemize}
\item typ typFeld[var]; typ *typPointer=typFeld (typPointer zeigt auf das Feld mit dem Index 0, typPointer+i zeigt auf das feld mit dem index i)
\item Dereferenzieren: *(typPointer+i)=var; oder typPointer[i]=var;(genau wie bei arrays)
\item array kann ohne arrayindex durchlaufen werden
\end{itemize}
\item immer vorsicht bei Pointerarithmetik
\item man kann pointer wie normale variablen casten
\item Casten muss immer mit Vorsicht durchgeführt werden
\end{itemize}
\subsubsection{Dynamische Speicherverwaltung}
\begin{itemize}
\item Arbeitsspeicher\subitem Stack(von dem PC verwaltet beinhaltet, Variablen/Parameter)\subitem Heap(wird durch den Entwickler verwaltet beinhaltet dynamische Datenstrukturen)
\item Bedarfsgerechte Nutzung des vorhanden Speichers, freigeben von nicht mehr benötigtem Speicher, verwendung für dynamische Datenstrukturen(variable Größe)\subitem\ensuremath{\Leftarrow} Effiziente Nutzung, Verwaltung liegt beim Entwickler
\item Heap starte bei der 1. Adresse, Stack startet bei der Letzten;
\item Speicherplatzanforderung:
\subitem malloc()\begin{itemize}
\item void *p; p=malloc(Anzahl\textunderscore Bytes);, liefert einen Pointer auf Speicherbereich der benötigten Größe, oder Null falls Speicherbereich der benötigten Größe nicht mehr vorhanden
\item Anzahl der Byte mittels sizeof(typ) ermitteln

\end{itemize}
\item free()\begin{itemize}
\item free(pointer); gibt den mit malloc reservierten speicherplatz frei(darf nicht mehrfach auf den selber verwendet werden)
\item freigegebene pointer dürfen nicht mehr dereferenziert werden
\item die Speicherplatzfreigabe sollte mit dem gleichem pointer wie die Anforderung erfolgen
\item nach der verwendung immer freigeben
\end{itemize}
\item Memory Leaks:\begin{itemize}
\item im Heap reservierter Speicherbereich  jedoch nicht mehr zugänglich (über Pointer erreichbar)
\item Probleme:
\subitem führt zu Speichermangel,
\subitem möglicherweise abnormale Programmbeendigung
\item Vermeidung
\subitem sorgfältige Programmierung Dynamischer Speicherverwaltung
\end{itemize}
\end{itemize}
\subsubsection{Funktionen}
\begin{itemize}
\item Deklaration: Ergenistyp funktionsname (Parameterleiste) {funktionsrumpf}
\item wenn keine ausgabe "void" als ergebnisstyp, sonst "return ergebnis;"(kann auch mehrmals vorkommen)
\item lokale Varaibeln, Konstante, Parameter sind nur innerhalb der Funktion bekannt, namensgleiche globale werden überschattet
\item funktionen möglichst klein halten (100zeilen) sinnvolle ausgaben (Rückgabe fehlercode) keine globalen variabeln in funktionen
\item übergabeprinzip: call by value beim aufruf werden die parameter durch kopieren ersetzt
\item parameter werden im stack bereich übergeben, nach schließen der funktion wird dieser frei
\item funktionen können auch rekursiv sein
\item es wird immer ein wert zurückgegeben
\item Rückgabe meherere Werte:\begin{itemize}
\item Rückgabe eines Typs
\item void functionname (typ *i){\dots return;} funktion schreibt direkt über die adressen in den speicher
\item Per Referenz
\end{itemize}
\item Funktionspointer\begin{itemize}
\item typ (*fPointer) (typ); zeigt auf einen funktion mit Returntyp typ und einem typ als Parameter
\item wie bei herkömmlichen pointer dereferenziert
\end{itemize}
\end{itemize}
\subsubsection{Dateien}
\begin{itemize}
\item Datentyp FILE ist in C vordefiniert
\item Deklaration: FILE *datei;
\subitem datei=fopen("c:\\dateiname.dateiendung","modus") \subitem modus: "r" Read "rb" ReadBinary "w" write "wb" write binary "a" Append "ab" append binary
\subitem fscanf(datei,\ensuremath{\text{"}< \%[Flag][Breite][\text{.PrÀzision}][\text{PrÀfix}]Typ>\text{"}}, zeile);
\item Binärer zugriff:\begin{itemize}
\item Lesen fread:\subitem fread(Puffer\textunderscore Adresse, Puffer\textunderscore Größe, Anzahl\textunderscore Puffer, FILE *datei)
\subitem Puffer\textunderscore Adresse=  Pointer auf dateibereich
\subitem Puffer\textunderscore Größe = Anzahl Bytes des Datenbereichs (sizeof())
\subitem Azahl\textunderscore Puffer = Anzahl der zu lesenden Records
\subitem datei = dateipuffer
\item schreiben fwrite(Puffer\textunderscore Adresse, Puffer\textunderscore Größe, Anzahl\textunderscore Puffer, FILE *datei)
\end{itemize}
\item fclose(File *datei); (nach zugriff immer schließen)
\end{itemize}
\end{document}
