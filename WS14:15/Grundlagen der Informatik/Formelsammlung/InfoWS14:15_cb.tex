
\documentclass[10pt,a5paper]{article}

\usepackage[latin5]{inputenc}
\usepackage{amsmath}
\usepackage{amsfonts}
\usepackage{amssymb}
\usepackage{graphicx}
\usepackage[left=0.05cm, right=1cm, top=1cm, bottom=1cm]{geometry}
\author{Christian Böhm}
\usepackage{color}
\definecolor{grey}{RGB}{211,211,211}
\newcommand{\cbg}[1]{\colorbox{grey}{#1}}

\begin{document}
\section{Information, Nachrichten}
\begin{itemize}
\item Informatik als Wissenschaft von der maschinelle Verarbeitung von Information
\subitem geeignete Darstellung von Information in einem Rechner
\subitem Eine abstrakte Information wird durch einen konkrete Nachrichten übermittelt. Einen Nachricht ist eine Sequenz von eichen aus einem vorgegebenen Zeichenvorrat.
\item Eine Nachricht kann verschieden Informationen enthalten und/oder unterschiedlich interpretiert werden, Sie kann an einen und mehrere Empfänger geschickt werden.
\item Es gibt eine Informationsvorschrift f, die mit dem Absender vereinbart ist
\item Nachrichtenträger:
\subitem Frequenz, Amplitude, Stromstärke, elektrische Spannung
\item der zeitliche Verlauf einer physikalischen Größe die eine Nachricht überträgt heißt Signal, die Kenngröße heißt Signalparameter
\subitem wenn der Signalparameter nur endlich viele Werte annehmen kann heißt das Signal digital
\item Ein Zeichen wird definiert als Element einer endlichen Menge unterscheidbarer "Dinge". Diese Menge wird als Zeichenvorrat bezeichnet
\subitem Ein Zeichenvorrad, der eine Reihenfolge hat wird als Alphabet bezeichnet
\end{itemize}
\section{Codierungen}
\begin{itemize}
\item Eine Vorschrift zur Abbildung eines Zeichenvorrats in einen andere Zeichenvorrat heißt Code
\item in technischen Codes werden Buchstaben, Ziffern und ggf. andere Zeichen fast immer durch Binärworte codiert
\subitem bei Computern werden meistens Erweiterungen von ASCII-Code verwendet (Alle ASCII Zeichen sind kompatibel)
\end{itemize}
\section{Betriebssysteme}
\begin{itemize}
\item Sammlung von Programmen zur Verwaltung der Hardware
\item Schichtmodell:
\subitem Schicht nutzt ausschließlich Dienste(Schnittstellen) der direkt darunter liegender Schicht, kein bereichsübergreifender Zugriff
\item Aufgaben \begin{itemize}
\item Verwaltung des Hauptspeichers
\item Verwaltung des Zugriffs auf der Peripherie
\item Steuerung des Ablaufs von Programmen
\item Zugriffsschutz
\item Abrechnungsinformation
\end{itemize}
\item Im Detail
\subitem Verwaltung einer Festplatte\begin{itemize}
\item Blockweise Übertragung von externem  Speicher vom/zum Rechner
\item Dateien belegen i.d.R. mehrere Blöcke auf der Festplatte (meist nicht konsekutiv)
\item Zusätzlich zum Inhalt wird für jede Datei weitere Information verwaltet
\item Anlegen Wiederfinden, Ändern, Löschen einer Datei
\item Verwalten der Dateiblöcke
\item Koordination der Plattenzugriffe (Minimale Bewegungen des Schreib-/Lesekopfs)
\end{itemize}
\subitem Prozessverwaltung:\begin{itemize}
\item Prozess: Eingenständer Ablauf eines Programms auf einem Rechner
\item Mehrprozesssysteme (Mehren Programme von einem oder mehreren Benutzen laufen Parallel auf dem selben Rechner)
\item Zuteilung von Ressourcen an Prozesse
\item Schutz der Prozesse voreinander vor gegenseitiger Störung
\item Zuteilung von Rechenzeit an Prozesse am Beispiel von Zeitscheiben (time slicing)
\item Prinzip:
\subitem Prozess bekomme den Prozessor exklusiv für einen bestimmten Zeitraum, wird der Prozessor einem anderen Prozess zugeteilt
\subitem quasi parallele Abläufe mehrere Prozesse auf einem Prozessor
\end{itemize}
\item Speicherverwaltung:\begin{itemize}
\item Mögliche Probleme: Prozesse greifen auf die selbe Speicherstelle zu und überschreiben gegenseitig Information
\item Jeder Prozess bekommt seinen eigenen Speicher, der physikalisch von den Speichern andere Prozesse getrennt wird(jeder Prozess hat den Eindruck den Hauptschuldiger exklusiv zu besitzen)
\item Speicher wird in Pages fester Größe aufgeteilt
\item In einer Tabelle wird jeder Seite eines Prozesses eine Seite im physikalischen Speicher(Kachel)
\item Jeder Prozess kennt nur eine logischen Adressen
\item Betriebssystem ordnet logische Adresse zu physikalischen Adressen zu
\item Speicher logisch größer als physikalisch
\item Seitengröße=Blockgröße der Festplatte
\item Jeder Seite des virtuellen Speichers wird ein Block der Festplatte zugeordnet
\item Seitentabelle gibt an ob und ggf. in welcher Kachel des phys. Speichers die Seite liegt, oder in welchem Block der Festplatte
\item benötigte Seite nicht im phys. Speicher \ensuremath{\Leftarrow} Holen von Platte, Auslagern einer nicht benötigten Seite auf Platte und Laden der benötigten in die frei gewordenen Kachel
\end{itemize}
\end{itemize}
\section{Rechnerarchitektur}
\subsection{Neumann Architektur}
\begin{itemize}
\item abstrakte, allgemein einsetzbare Rechnerarchitektur
\item wird von heutigen Rechner umgesetzt
\item abstrakte Sichtweise
\item dient dem grundlegenden Verständnis der Funktionsweise eines Computers und der Abarbeitung von Software
\item zur Lösung eines Problems muss von außen ein Programm eingegeben und im Speicher abgelegt werden
\item Programme, Daten inkl. Zwischen- und Endergebnisse nutzen den Selben Speicher
\item Speicher inkl. in gleich große Zellen unterteilt, die fortlaufend nummeriert sind
\item über die Adresse eine Speicherzelle kann deren Inhalt gelesen oder geschrieben werden
\item Ausführung eines Programms wird durch das Steuerwerk gesteuert
\item Aufeinander folgende Befehle eines Programms werden in aufeinanderfolgenden Speicherzellen abgelegt
\item das Ansprechen des nächtens Befehls geschieht vom Steuerwerk durch erhöhen der Befehlsadressen um eins
\item durch Sprungbefehl kann von der Bearbeitung der befehle in der gespeicherten Reihenfolge
\item Es gibt mindestens:\begin{itemize}
\item arithmetische Befehle wie Vergleiche, logisches und, oder usw.
\item Transportbefehle, z. B. von Speicher zum Rechenwert und für die Ein-/Ausgabe
\item bedingte Sprünge
\end{itemize}
\item Der Speicher:\begin{itemize}
\item für den Zugriff auf den Speicher existieren zwei Leitungen(Busse): Adressbus und Datenbus
\item Jede Speicherzelle umfasst "Wortbreite" Bits
\item Auf dem Adressbus wird die Adresse "Nummer" der Speicherstelle übermittelt, auf die zugegriffen werden soll
\item soll im Speicher geschrieben werden, wird die adressierte Speicherzelle mit dem Wert auf dem Datenbus überschrieben
\item soll im Speicher gelesen werden, wird der wert der adressierten Speicherstelle auf den Datenbus geschriebenen
\end{itemize}
\item Das Steuerwerk:\begin{itemize}
\item steuert die Ausführung eines Programms
\item hold Befehle nacheinander aus dem Speicher(fetch) in das Befehlsregister
\item Jeder Programmbefehle besteht aus zwei Teilen:
\subitem Der Operationsteil legt fest, was gemacht werden soll
\subitem der Adressteil bestimmt, auf welche Daten der Befehl anzuwenden ist
\item jeder Programmbefehl wird decodiert (decode)
\item danach ausgeführt (execute) in dem entsprechen Funktionseinheiten(z.B Rechenwerk , Ein-/Ausgabe aktiviert werden)
\item Das Befehlszählerregister erhöht die Speicheradresse des aktuell ausgeführten Befehls
\item Nach Ausführung eines Befehls wird das Befehlsregister geändert und der nächsten Befehl wird geholt
\subitem  Befehlsregister +1( falls kein Sprungbefehl)
\subitem Sprungziel sonst
\item Sind Daten für eine Befehlsausführung im Speicher abgelegt, so müssen diese vor Ausführung geholt bzw nach Ausführung gespeichert werden:
\subitem zuerst wird bestimmt auf welche Speicherstelle zugegriffen werden muss (Adressbrechung: Adresse)
\subitem dann wird der Speicherzugriff(read/write durchgeführt)
\item Vollständiger Befehlsfluss
\subitem \ensuremath{fetch\Leftarrow decode\Leftarrow address\Leftarrow read\Leftarrow execute \Leftarrow write}
\end{itemize}
\item Das Rechenwerk:\begin{itemize}
\item Akkumulator dient als Speicher für Zwischenergebnisse
\item Beim klassischen Von-Neumann-Modell enthält der Akkumulator immer den ersten Operanden (evlt. vorher aus dem Speicher laden!) und das Ergebnis
\item Die Rechenlogik führt arithmetischen Operationen aus
\end{itemize}
\end{itemize}
\subsection{Reale Rechnerarchitekturen}
\begin{itemize}
\item Laden größerer Speicherbereiche in speziellen Speicher (Cache)
\item Bearbeitung mehrere Befehle gleichzeitig (Pipelining: Parallelität innerhalb der Befehlsausführung durch Überlagerung von Phasen aufeinanderfolgender Befehle)
\item Befehlszyklus ohne Pipelining:
\subitem \ensuremath{fetch\Leftarrow decode\Leftarrow address\Leftarrow read\Leftarrow execute \Leftarrow write}
\item Befehlszyklus mit Pipelining:
\subitem gleichzeitiges Nutzen "disjunkter" Teile des Rechenwerks (Ganzzahlarithmetik/Gleitkommaarithmetik)
\item Speicherhierarchien
\subitem sehr schnelle Cache Speicher
\subitem "langsamere" Hauptspeicher
\subitem "Langsamer" Plattenspeicher
\item Parallele Prozessoren mit gemeinsamen Speicher (shared memory)
\item Parallele Prozessoren mit verteilen Speicher (distributed memory)
\end{itemize}
\section{Zahlensysteme}
\begin{itemize}
\item Dezimal(Basis 10)\begin{itemize}
\item \ensuremath{d=d_{n-1}*10^{n-1}+\dots+d_0}
\subitem \ensuremath{d_i\in \{0,\dots,9\}}
\end{itemize}
\item Basis n \begin{itemize}
\item \item \ensuremath{z=z_{n-1}*b^{n-1}+\dots+z_0}
\subitem \ensuremath{z_i\in \{0,\dots,b-1\}}; \ensuremath{b\ge 2}
\end{itemize}
\item Praktisch relevante Basen:
\subitem b = 10 (Dezimal) \ensuremath{z_{10}}
\subitem b = 2  (Binär-/Dual) \ensuremath{z_2}
\subitem b = 16 (Hexadezimal) \ensuremath{z_{16}}
\subitem b = 8 (Oktal) \ensuremath{z_8}
\subitem b = 1 (Unäre) \ensuremath{z_1} (Strichliste)
\end{itemize}
\subsection{Binärzahlen}
\begin{itemize}
\item Rechen mit Binärzahlen\begin{itemize}
\item Addition
\subitem \ensuremath{0+0=1;0+1=1;1+1=10}
\item Subtraktion
\subitem \ensuremath{0-0=0;1-0=1;1-1=0;0-1=10(\text{Übertrag})}
\end{itemize}
\item Umwandlung \begin{itemize}
\item \ensuremath{Hexadezimal\Leftarrow \text{Binär}}
\subitem Eine Hexadezimalstelle kann durch 4 Binärstellen ausgedrückt werden
\item \ensuremath{Oktal\Leftarrow \text{Binär}}
\subitem Eine Oktalstelle kann durch 3 Binärstellen
\item BCD(Binäry coded Dezimal)
\subitem jeder Dezimalziffer wird mit 4 Binärziffer dargestellt
\end{itemize}
\item Negative Zahlen\begin{itemize}
\item Explizite Speicherung des Vorzeichens(+ durch 0, - durch 1)
\item nur bei vorzeichenbehafteten Größen:
\subitem Letzter Bit(Binär) steht für das Vorzeichen (+ durch 0, - durch 1)
\item Komplementbildung
\subitem Einerkomplementdarstellung (Invertieren Aller Bits) (\ensuremath{-0}(negative Null) addieren)
\subitem Zweierkomlementdarfstellung (Invertieren aller Bits und 1 addieren)
\end{itemize}
\item Ganzzahlarithmetik\begin{itemize}
\item Schiebe Operatoren(shifts): \begin{itemize}
\item \ensuremath{<< \text{ Linksshift, } >> \text{ Rechtsshift}}
\item Linksshift ziehen immer einen "0" nach, Rechtsshift auf unsinged Größen ebenfalls
\item Rechtsshift auf auf singed Größen ziehen das Vorzeichenbit nach
\item Multiplikation bzw. Division durch 2er Potenzen (bei Laufzeitkritischen Anwendungen)
\end{itemize}
\end{itemize}
\item Fixed Point
\subitem PRO: Rechnung schnell
\subitem CONTRA: Rundungsfehler
\item Floating Point (\ensuremath{Gz:=(<Vorzeichen>)<Mantisse>*Basis^{<Vorzeichen>Exponent}})
\subitem Eine Gleitkommazahl heißt normiert,wenn die Mantisse \ensuremath{<1} und die erste Nachkommastelle ungleich Null ist
\end{itemize}
\section{Aussagenlogik}
\begin{itemize}
\item Eine Aussage ist ein Objekt, dem sich eindeutig ein Wahrheitswert (true,T)/(false,F) zuordnen lässt
\item Operatoren\begin{itemize}
\item Konjunktion \ensuremath{\land}
\subitem Gesamtaussage ist wahr, wenn A und B wahr sind
\item Disjunktion \ensuremath{\lor}
\subitem Gesamtaussage ist wahr, wenn mindestens eine Aussage wahr ist
\item Negation \ensuremath{\lnot}
\subitem Ergebnis falsch, wenn A wahr und umgekehrt
\item Implikation \ensuremath{\implies}
\subitem Ergebnis whar, wenn entweder A falsch oder A und B wahr
\item Äquivalenz \ensuremath{\iff}
\subitem Ergebnis wahr, wenn entweder A und B Falsch oder A und B wahr
\item Exklusives oder \ensuremath{xor}
\subitem Ergebnis war, wenn A und B ungleich
\item negiertes oder \ensuremath{nor}
\subitem Ergebnis wahr, wenn A und B unwahr
\item negiertes und \ensuremath{nand}
\subitem Ergebnis wahr, wenn A oder B unwahr
\end{itemize}
\item Eine Formel ist erfüllbar, wenn mindestens eine Belegung gibt, deren Auswertung den Wert wahr liefert
\item Eine Formel heißt Tautologie, wenn jeder Belegung den Wert wahr liefert
\item Eine Formel heißt unerfüllbar (Kontradiktion) wenn es keine Belegung gibt, die die Formel erfüllt
\item Zwei Formeln heißen äquivalent(\ensuremath{\tilde{=}}) wenn jede Belegung der Formlen zum gleichen Wert führt
\item Äquivalenzen:\begin{itemize}
\item Kommutativität
\subitem \ensuremath{F_1\land F_2\tilde{=}F_2\land F_1} oder \ensuremath{F_2\land F_1\tilde{=}F_1\land F_2}
\item Assoziativität
\subitem \ensuremath{(F_1\land F_2)\land F_3 \tilde{=}F_1\land (F_2\land F_3)} oder \ensuremath{(F_1\lor F_2)\lor F_3 \lor{=}F_1\lor (F_2\lor F_3)}
\item Distributivität
\subitem \ensuremath{(F_1\land F_2)\lor F_3 \tilde{=}(F_1\lor F_3)\land(F_2\lor F_3) } oder \ensuremath{(F_1\lor F_2)\land F_3 \tilde{=}(F_1\land F_3)\lor(F_2\land F_3) }
\item Idempotenz
\subitem \ensuremath{F_1\land F_1\tilde{=}F_1} oder \ensuremath{F_1\lor F_1\tilde{=}F_1}
\item De Morgan'sche Regel
\subitem \ensuremath{\lnot (F_1\land F_2)\tilde{=}\lnot F_1\lor \lnot F_2} oder \ensuremath{\lnot\lnot (F_1\lor F_2)\tilde{=}\lnot F_1\land \lnot F_2}
\item Implitaktionsregel
\subitem \ensuremath{F_1\iff F_2\tilde{=}\not F_1\lor F_2}
\end{itemize}
\item Konjunktive Normalform
\subitem Eine Formel F ist in konjunktiver Normalform (KNF) , wenn sie eine Konjunktion von Disjunktionen von Literalen ist
\subitem Ein Literal ist dabei eine atomare Formel oder die Negation eine atomaren Formel (positives bzw. negatives Literal)
\subitem \ensuremath{F=\land^k_{i=1}(\lor^{mk}_{j=1}L_ij)}
\item Disjunktive Normalform
\subitem Eine Formel F ist in disjunktiver Normalform (DNF), wenn sie eine Disjunktion von Konjunktionen von Literalen ist
\subitem \ensuremath{F=\lor^k_{i=1}(\land^{mk}_{j=1}L_ij)}
\item Für jeder Formel F gilt:
\subitem zu F gibt es eine äquivalente Formel in konjunktiver Normalform und eine äquivalente in Disjunktiver Normalform
\item deMorgan'sche Verknüpfungsbase n\ensuremath{\{\lor,\lnot\}};\ensuremath{\{\land,\lnot\}}
\item Boole'sche Verknopfungsbasis \ensuremath{\{\land,\lor,\lnot\}}
\item Frege-Basis \ensuremath{\{\lnot, \iff\}}
\item nand-Basis \ensuremath{\{nand\}}
\item nor-basis \ensuremath{\{nor\}}
\end{itemize}
\section{Prädikatenlogik}
\begin{itemize}
\item Sei \ensuremath{M} eine menge (Gegenstandsbereich, INdividuenbereich). Dann heißst einen Abblunding \ensuremath{M^n\leftarrow \{wahr,falsch\}} ein n-stelliges Prädikat über der Menge M
\item Logische Operatoren\begin{itemize}
\item Konjunktion \ensuremath{\land}
\subitem Gesamtaussage ist wahr, wenn A und B wahr sind
\item Disjunktion \ensuremath{\lor}
\subitem Gesamtaussage ist wahr, wenn mindestens eine Aussage wahr ist
\item Negation \ensuremath{\lnot}
\subitem Ergebnis falsch, wenn A wahr und umgekehrt
\item Implikation \ensuremath{\implies}
\subitem Ergebnis wahr, wenn entweder A falsch oder A und B wahr
\item Äquivalenz \ensuremath{\iff}
\subitem Ergebnis wahr, wenn entweder A und B Falsch oder A und B wahr
\end{itemize}
\item Quantoren:
\subitem \ensuremath{\forall} heißt Allquantor (für alle)
\subitem \ensuremath{\exists} heißt Existenzquantor (existiert)
\item Variablen, die an Quantoren gebunden sind, heißen gebundene Variablen, alle anderen heißen freie Variablen
\item Eine Formel ohne konstante Prädikate, die für jede Wahl der variablen Prädikate und dann fpr jede mögliche Wahl der Individuenvariablen immer den Wert T ergibt, heißt Gesetz der Prädikatenlogik
\end{itemize}
\section{Zustandsautomaten}
\begin{itemize}
\item Mealy-Automat\begin{itemize}
\item Ein Mealy-Automat ist durch \ensuremath{(Z,z_0,E,A,T)} gegeben
\subitem \ensuremath{Z} ist eine endliche nichtleere Menge von Zuständen
\subitem \ensuremath{z_0, z_0\in Z} ist der Anfangszustand
\subitem \ensuremath{E} ist die endliche Menge der möglichen Eingaben
\subitem \ensuremath{A} ist die endliche Menge der möglichen Ausgaben
\subitem \ensuremath{T} ist die Menge der Transitionen(Zustandsübergänge)
\item Jeder Transition \ensuremath{t\in T} ordnen einem Ausgangszustand \ensuremath{z_a\in Z} und einer Eingabe \ensuremath{e\in E} einen Folgezustand \ensuremath{z_f \in Z} und einen Ausgabe \ensuremath{a\in A} zu:\ensuremath{(z_a,e)\to (z_f,a)}
\end{itemize}
\item Graphische Darstellung:\begin{itemize}
\item Jedem Zustand wird ein Knoten (gezeichnet als Kreis) zugeordnet.
\item jeder Knoten wird mit dem Namen des zugehörigen Zustands beschriftet
\item jede Transition wird einen Gerichtete kante (gezeichnet als Pfeil) von Zustand A nach zustand F zugeordnet
\item Jeder Transition wird mit ihrer Eingabe e und Ausgabe a beschriftet
\end{itemize}
\end{itemize}
\section{Modularisierung}
\begin{itemize}
\item Zerlegung in unabhängig voneinander zu erstellende SW-Teile (Module)
\item Module sind getrennt voneinander übersetzbar
\item Grundlage der Zerlegung ist eine vor der Codierung definierte SW Architektur
\item Die Gesamtfunktion wird durch das "Zusammenspiel" der einzelnen Module realisiert
\item Elemente eines Moduls:
\subitem Konstanten, Variablen, Funktionen und Prozeduren werden in anderen Modulen verwendet
\item Prinzipien:
\subitem per \#define definierte Konstanten/Makros werden in anderen Modulen verwendet
\item Realisierung in C:\begin{itemize}
\item per \#define definierte Größen werden in einem Headerfile (.h) definiert, dieses diese werden, wo benötigt, eingebunden(\#include)
\item Eine Konstante(const) oder Variable wird in einer Datei(.c) global definiert. Über eine externe Deklaration in einem Headerfile kann diese in anderen Modulen bekanntgemacht werden
\subitem Selbiges gilt auch für Funktionen/Prozeduren
\end{itemize}
\item Globale Namen müssen eindeutig sein
\item Ein Modul muss das Hauptprogramm enthalten
\item Verwendung von Funktionen und Prozeduren modulübergreifend ist OK
\item globale Variablen sind zu vermeiden(Informationen möglichst über Schnittstelle austauschen)
\end{itemize}
\newpage
\section{Prozessmodelle}
\begin{itemize}
\item Phasenmodelle:\begin{itemize}
\item Wasserfall-Modell
\item Prototyping, evolutionäre Entwicklung
\item Formale Entwicklung
\item Wiederverwendungsorienttierete Entwicklung (opportunistisch/strategisch)
\item Prozesswiederholende Modelle (inkrement/agile)
\item Modellbasierte Entwicklung
\item V-Modell

\end{itemize}
\item Wasserfall-Modell:\begin{itemize}
\item Strikte Reihenfolge von Projektphasen
\item Korrekturzyklen werde nur auf die vorangehende Phase beschränkt
\item Vollständiges Ausführen einer Phase, Dokumentation als Abschluss
\item Nachfolgephase beginnt erst nach vollständigem Abschluss einer Phase
\item Auftraggeber und Anwender nur in der Phasenplanung und bei der Abnahme beteiligt
\item Vorteile:
\subitem Einfaches und klares Vorgehen
\subitem Relativ einfach umzusetzen (Planung/Steuerung)
\subitem Erzwingt eine gewisse Disziplin bei den Entwicklern
\item Nachteile:
\subitem Entspricht nicht der Realität
\subitem Phasen können sich gegenseitig bedingen
\subitem schwache Einbindung des Auftraggebers, bzw. der Anwenders
\subitem Spätes Erkenne von Risiken, die sich aus der Implementierung ergeben
\subitem Spätes Erkennen von Fehler aus den frühen Phasen, die erst in späteren Phasen erkennbar werden
\subitem Dokumentation bekommt einen manchmal zu hohen Stellenwert
\subitem Zeitverzug aus frühen Phasen schlägt voll durch auf spätere Phasen
\subitem Im Extremfall ist das Produkt nicht brauchbar, die Entwicklung muss neu von vorne begonnen werden
\subitem Säte Änderung und Fehlerbehebung im Code führen zu veralteter Dokumentation
\end{itemize}\newpage
\item V-Modell:\begin{itemize}
\item Verbessertes Wasserfallmodell
\subitem ergänzt um QS in allen Phasen, standardisiert
\item Ziel: Entwurfsfehler frühzeitig zu erkennen
\item Aber Nachteile des Wasserfallmodels bleiben erhalten:
\subitem starre Reihenfolge der Phasen, Ergebnis in einem Durchlauf entwickelt
\end{itemize}
\item V-Modell-XT:\begin{itemize}
\item Verbesserung der Unterstützung von Anpassbarkeit, Anwendbarkeit, Skalierbarkeit und Änder- und Erweiterbarkeit des V-Modells
\item Berücksichtigung des neusten stand der Technologie und Anpassung an aktuelle Vorschriften und Normen
\item Erweiterung des Anwendungsbereiches auf die Betrachtung des gesamten Systemlebenszyklus, im Rahmen von Entwicklungsprojekten
\item einführen eines organisationsspezifischen Vierbesserungsprozesses für Vorgehensmodelle
\end{itemize}
\end{itemize}

\end{document}
